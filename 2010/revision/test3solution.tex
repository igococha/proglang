\documentclass{article}

\usepackage{a4wide}
\usepackage[breaklinks=true,
        colorlinks=true,urlcolor=blue,pdfpagemode=None]{hyperref}

\newcommand{\fname}[1]{\texttt{#1}}
\newcommand{\regla}[1]{$\;\mathbf{\stackrel{{}_{#1}}{\longrightarrow}}\;$}
%\newcommand{\regla}[1]{$\mathbf{\rightarrow^{#1}}$}

\newcommand{\comment}{\textbf{comment}}
\newcommand{\cuatro}{$\;\;\;\;$}

\begin{document}
\thispagestyle{empty}

\newcommand{\negra}[1]{\textbf{#1}}

%%%%%%%%%%%%%%%%%%%%%%%%%%%%%%%%%%%%%%%%%%%%

This document contains the solution for Test 3. I have added additional comments, usually preceded by the keyword \comment.


\section*{Test 3 - Solution}

\begin{enumerate}

\item The annotated SPL code given in question 1 gives us enough information the deduce the typechecking rules of the modulus operator.
\begin{itemize}
\item We are told that \verb?x := (y+2) % 5? is correctly typed. A quick inspection to the right-hand side of the assignment (an expression that uses the modulus operator) tells us that both operands, \verb?(y+2)? and \verb?5?, are of type integer. 
\item Also, the modulus operator must return an integer since an assignment requires that the types of the assigned variable and the expression (right-hand) are the same.
\item The next two lines suggest that the modulus operator does not accept float operands.
\item The final line can be used to verify that the modulus operator returns an int. Typechecking fails because the assigned variable is a float.
\end{itemize}

Answer:\\
\begin{tabular}{l}
typecheck(Modulus(e1,e2),f,stable) = \\
\cuatro t1 = typecheck(e1,f,stable) \\
\cuatro ReportError if (t1 != int) \\
\cuatro t2 = typecheck(e2,f,stable) \\
\cuatro ReportError if (t2 != int) \\
\cuatro return int
\end{tabular}

that is, e1 and e2 must be correctly typed, and their type must be int. The modulus operator (as defined in this example) returns an integer. Could we have a version that accepts floats?

\item Typechecking.

The errors are:
\begin{itemize}
\item Line 7: The type of the right-hand side of the assignment must match the type of the assigned variable. In this case, the type of the right-hand side is float (the addition of two floats returns a float) but the assigned variable is an int.
\item Line 9: The function declaration of g(int,float) indicates that it requires two parameters: an int and a float. However, the function call g(x,y) passes two integers - the second argument has the wrong type.\\
\comment: Note the return type of g, boolean, matches the type of the left-hand side of the assignment (the type of variable b).
\item Line 11: The + operation requires both operands to be of the same type, int or float.
\item Line 15: Undeclared variable z.
\end{itemize}

\comment: Line 10 is correct. First of all, the test/condition in an IF statement
 must be an expression. Variables are expressions so the syntax of line 10 is correct. Furthermore, the type of the test must be boolean: the type of expression \textbf{b} is boolean as well.


\item Stack frames

Given the following SPL program extract:
\begin{verbatim}
 int fone(int x, float y) {
   int a; int b;
   float c;
   // body is not important
   d := ftwo((float) c, y, 15.5);
   // more body comes here
  return (float) c;
}

boolean ftwo(float p1, int p2, float p2) {
  // body is not important
}
\end{verbatim}

\comment: Note that I have added two (float) type casts to fix the type errors.

\begin{itemize}
\item[a.] The stack frame layout of function fone is:

\begin{tabular}{lll}
offset & content & type(size) \\
\hline 
0 & local a & int(1) \\
1 & local b & int(1) \\
2 & local c & float(2) \\
4 & local d & boolean(1) \\
5 & return address & address(1) \\
6 & returned value & boolean(1) \\
7 & ftwo param p3 & float(2) \\
9 & ftwo param p2 & int(1) \\
10 & ftwo param p1 & float(2)
\end{tabular}

Frame size = 12

\comment: A function's stack frame (assuming that it does not have to store any temporaries or values stored in registers) stores:
\begin{itemize}
\item  its local variables (fone's a,b,c,and d)
\item  If it makes any function calls (in our example, fone calls ftwo): 
\begin{itemize}
\item The return address. This is the address of the instruction where execution resumes after the call to ftwo terminates i.e. the address of the instruction inmediately after \verb+d := ftwo((float) c, y, 15.5)+. The virtual machine needs to know where to return! The size of this is always one (the size of an address).
\item The value returned by the called function (ftwo's returned value, a boolean). The size of this depends on the size of the return type of the called function.
\item The parameters passed to the called function (ftwo's parameters p1,p2 and p3)
\end{itemize}
\end{itemize}

\item[b.] How does fone access its parameter y? Write TPL code that stores 10.5 into y.

Parameters x and y are located above fone's frame pointer (FP). They have offsets -1 and -3, respectively. The global location of y is FP(-3). If we want to store 10.5 into y then we have to write:

STORE 10.5, FP(-3)

\comment: The space to store fone's x and y is part of the stack frame of fone's caller (not shown here since this is just a program extract).

\end{itemize}



\end{enumerate}
%%%%%% end of test 3

\end{document}

