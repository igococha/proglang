\documentclass[11pt]{article}
\usepackage{a4wide}
\usepackage{semantic}
\usepackage{verbatim}

\usepackage{color}

\usepackage[breaklinks=true,colorlinks=true,urlcolor=cyan]{hyperref}

\reservestyle{\nonterm}{\textit}
\nonterm{Statement,Exp,id,Case}

\reservestyle{\keyword}{\textbf}
\keyword{var,function,let,in,end,if,then,else,while,do,for,break,to,switch,of,case,default}

\thispagestyle{empty}

\begin{document}

\subsection*{Language Processors Coursework 1}

\paragraph{Organisation:}
Do this work in pairs if you wish. You will hand
work in as individuals but the handin procedure will allow 
you to say who you collaborated with. 
\emph{Pairs} only -- not threesomes!

\paragraph{Handin:} 
The deadline and the electronic handin procedure, and exactly what you should 
hand-in, are documented online.
Obviously you should change file and directory permissions while you 
are working so that your work is not visible to others -- 
remember plagiarism carries penalties.


\begin{enumerate}
\item

Use regular expressions to define precisely signed floating
 point numbers.
 Such numbers must contain a decimal point, and at least one
 digit before and after the decimal point.
 They may optionally be followed by a signed exponent
 that begins with the letter `\verb"E"' and is followed
 by a (possibly signed) integer.
 There should be no leading zeros before the decimal point or the
 exponent integer, and no trailing zeros after the decimal point.
 In this notation, \verb"39.37", \verb"-6.336E4", \verb"0.894E-4" and
 \verb"0.0" are legal, while \verb".36", \verb"4.",
 \verb"+.7E6",  \verb+01.7+, \verb+1.70+ and \verb+1.7E04+
 are illegal.
 (Note: these are not Java floating point numbers; the Java definition
 is more complex than this.)
Implement and test your expressions using JavaCC (make your
	expressions readable and understandable).

\item Copy the MiniJava implementation directory
	to where you want to work, using:
\begin{verbatim}
   cp -R /soi/sw/courses/daveb/IN2009/minijava/chap4 
\end{verbatim}
({\em OR}\/ download it from WebCT).
Files README in the various directories
give a brief
description of the structure of the implementation.

\begin{enumerate}
\item Add a Java-like for-statement to the MiniJava 
	implementation.
\[
\<Statement> \; -> \; \<for> \; ( \<id> = \<Exp> \; ; \; \<Exp> \; ; \; \<id> = \<Exp> ) \; \<Statement>
\]
%	The token definitions and abstract syntax tree code
%	for it is included in the \verb+mjabs+ implementation directories.
	You will need to add the for-statement to the JavaCC specification
        (including appropriately adjusting the token regular expressions), 
	so that it builds the correct abstract syntax trees,
        and write new appropriate abstract syntax tree classes.
        The pretty-printer will also
        need to be updated to appropriately print the for statement 
        abstract syntax you have introduced.
%\item The short-circuit boolean operators \verb+&+ and \verb+|+ have
	%also been removed from the CUP specification grammar.
	%They are not represented in the abstract syntax as
	%\verb+OpExp+s but instead should be represented as \verb+IfExp+
	%nodes (see Appel page 105). Put these operators back in
	%the grammar so that it builds the correct abstract syntax trees
	%for them.
\item In a similar way, 
      add the Java-like switch-statement defined below to the implementation:
\[
\;\;\;\;\;\;\;\;\;\<Statement> \; -> \; \<switch> \; ( \; \<Exp> \; ) \; \{ \; \<Case>^{*} \; \}
\]
\[
-> \; \<break> \; ;
\]
\[
\<Case> \; -> \; \<case> \; \<Exp> \; : \; \<Statement>^{*}
\]
\[
\;\;\;\;\;\;\;\; -> \; \<default> \; : \; \<Statement>^{*}
\]
Note that the $\<Case>$s may be represented as a list in your abstract syntax
(and hence programmed like the other lists).
\end{enumerate}

\end{enumerate}

\bigskip\noindent
{\em Note: remember that these exercises deal only with lexical and 
syntax analysis and producing appropriate abstract syntax trees: you do
not yet have to worry about how MiniJava programs are type-checked,
execute or have code produced for them!}

\end{document}
