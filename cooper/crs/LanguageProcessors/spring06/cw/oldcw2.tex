\documentclass[11pt]{article}
\usepackage{a4wide}
\usepackage{verbatim}

\usepackage{color}

\begin{document}

\subsection*{Language Processors Coursework 2}

\paragraph{Organisation:}
Do this work in pairs if you wish. You will hand
work in as individuals but the handin procedure will allow 
you to say who you collaborated with. 
\emph{Pairs} only -- not threesomes!

\paragraph{Handin:} 
The deadline and the electronic handin procedure are
on WebCT.
Obviously you should change file and directory permissions so 
that your work is not visible to others -- 
remember plagiarism carries penalties. 

\subsubsection*{The work}

You will extend the MiniJava implementation, including
the typechecker. Copy the MiniJava implementation directory
to where you want to work, using:  
\begin{verbatim}
   cp -R /soi/sw/courses/daveb/IN2009/minijava/chap5 .
\end{verbatim}
This is available as a zip file on WebCT too.
Files README in the various directories give a brief
description of the structure of the implementation,
and we studied the symbol table mechanism and 
typechecker in Session 5.

\begin{enumerate}
\item Add the operators \verb+/+ (divide), \verb+>+ (greater than),
\verb+||+ (or), and \verb+?:+ (shortcut if-else) to your implementation.
They have the same syntax and meaning as they do in Java, 
and appropriate typechecks should be implemented for them
(\verb+op1?op2:op3+ returns \verb+op2+ if \verb+op1+ is true,
and returns \verb+op3+ if \verb+op1+ is false).


\item Add appropriate typechecking for the for and switch statements
you implemented in the first coursework (for those who did
not handin the first coursework, I will publish some
guidance on what needs to be added).
\end{enumerate}

\end{document}
