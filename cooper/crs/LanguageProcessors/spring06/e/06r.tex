%\documentstyle[cityexam,11pt]{article}
\documentclass[11pt]{bareexam}
\usepackage{semantic}
\withmodelanswers
\begin{document}
%\examnumber{???}
%\begin{preamble}
%\degrees{???}

%\part{II}
%\title{Language Processors}
%\examdate{???}
%\examtime{???}
%\rubric{Answer {\sl TWO} questions \\ All questions carry equal marks}
%\end{preamble}
%\externals{???}
\internals{\ }

\reservestyle{\nonterm}{\textit}
\nonterm{Statement,Exp,dec,decs,vardec,fundec,id,exp,tyfields,type-id,fundecs,expseq,op}
\reservestyle{\keyword}{\textbf}
\keyword{var,function,let,in,end,if,then,else,while,do,for,break,to,switch,of,case,default}


\begin{questions}

\question

\begin{subquestions}

\subquestion

What is an ambiguous grammar? Show that the following grammar is ambiguous:

\begin{quote}
\begin{tabbing}
stmtxxx\=$\rightarrow$xxx\=if\kill
\it
S \> $\rightarrow$ \> \textbf{if} {\it E\/} \textbf{then} {\it S} \\
\it
     \> $\mid$ \> \textbf{if} {\it E\/} \textbf{then} {\it S\/} \textbf{else} {\it S} \\
     \> $\mid$ \> {\it other}
\end{tabbing}
\end{quote}
Here {\it other\/} stands for any other statement {\it S}.
\marks{15}

\begin{modelanswer}
Two parse trees for \verb"if a then if b then s1 else s2".
\end{modelanswer}


\subquestion

The reference manual for a MiniJava-like programming language contains
the following grammar rule for an if-statement: 

\[
\<Statement> \; -> \;\; \<if> \;\; ( \; \<Exp> \; ) \;\; \<Statement> \;\; \<else> \;\; \<Statement> 
\]


\begin{subsubquestions}

\subsubquestion
Draw or write down a possible abstract syntax for the if-statement.
\marks{10}

\begin{modelanswer}
\begin{verbatim}
If Exp Statement Statement
(be flexible about how this is expressed - it might be Java and have types)
\end{verbatim}
\end{modelanswer}

\subsubquestion
Show how semantic actions in a grammar for a parser-generator such as JavaCC
can be used to produce abstract syntax trees for the if-statement. 
\marks{25}

\begin{modelanswer}
\begin{verbatim}
Statement IfStatement() :
{ Exp e;
  Statement s1,s2;
}
{
  "if" "(" e=Expression() ")" s1=Statement() "else" s2=Statement() 
  { return new If(e,s1,s2); }
}
\end{verbatim}
5 marks for framework, 10 marks for syntax, 10 marks for correct actions.
\end{modelanswer}

\subsubquestion
Informally describe an appropriate typecheck for the if-statement.
\marks{10}

\begin{modelanswer}
e must be a boolean expression.
\end{modelanswer}

\subsubquestion

Suppose a compiler for a MiniJava-like language that includes
an if-statement translates all statements and expressions
into intermediate code, for example intermediate representation (IR) trees.
Outline the intermediate code that might be generated
in translation of the if-statement. You may wish to use a simple
example to explain your translation, eg:
\begin{quote}
\begin{verbatim}
if (a < b) c = a; else c = b;
\end{verbatim}
\end{quote}
You can assume that the expression tree for any variable \verb"v" is
simply \verb"TEMP v". 
\marks{40}

\begin{modelanswer}
\begin{verbatim}
SEQ(SEQ(CJUMP(LT, TEMP a, TEMP b, L0, L1),
        SEQ(SEQ(SEQ(LABEL L0,
                  MOVE(TEMP c, TEMP a)),
                JUMP(NAME L2)),
            SEQ(SEQ(LABEL L1,
                  MOVE(TEMP c, TEMP b)),
                JUMP(NAME L2)))),
     LABEL L2)
\end{verbatim}
(might be a longer translation where a < b is evaluated to 
the 0 or 1 and then checked)

Might be expressed as trees.

15 marks for the comparison, 15 marks for appropriate jumping code, 10
marks for the assignments code.
\end{modelanswer}

\end{subsubquestions}

\end{subquestions}

\newpage

\question

\begin{subquestions}

\subquestion
The following regular expression recognises certain strings over the
alphabet $\{a,b,c\}$
\[
a(b|(cb))\!*c*
\]
Indicate which of these five strings are recognised by the above regular expression:
\begin{quote}
$acc$, $abac$, $a$, $abcbcbccc$, $abbbccbc$
\end{quote}
Also, write down three more strings that are recognised by the above expression.
Finally, write down two more strings consisting of 
the letters $a$, $b$ and $c$ that are \emph{not} 
recognised by the above regular expression.
\marks{25}

\begin{modelanswer}
Yes, No, Yes, Yes, No. 3 marks each. 
Five further strings, 2 marks each.
\end{modelanswer}

\subquestion

Explain why left-recursion must be eliminated from
grammar productions which are to be used in
construction of a recursive-descent parser.
Write down a general rule for rewriting left-recursive
grammar productions to be right-recursive and use it to 
rewrite the following productions to be right-recursive:
\begin{quote}
\begin{tabbing}
stmtxxx\=$\rightarrow$xxx\=if\kill
\it
S \> $\rightarrow$ \> ( {\it L\/} ) \ \  $\mid$ \ \  {\it a} \\
\it
L    \> $\rightarrow$ \> {\it L\/} , {\it S\/} \ \  $\mid$ \ \  {\it S} \\
\end{tabbing}
\end{quote}
\vspace{-1em}
\marks{45}

\begin{modelanswer}
Because the usual way of writing the 
procedures leads to immediate recursive call. 5 marks.

Rule (15 marks):
\begin{verbatim}
A  -> A a | b
(a and b strings of terms and non-terms)
rewrites to 
A  -> b A'
A' -> a A' | empty
\end{verbatim}
Answer (25 marks):
\begin{verbatim}
S  -> ( L ) | a
L  -> S L'
L' -> , S L' | empty
\end{verbatim}
\end{modelanswer}

\subquestion
Consider the following Java class:
\begin{verbatim}
1  class A {
2   String a; int c;
3   public void f(int b, String c) {
4     System.out.println(c);
5     int d = 3;
6     int a = b;
7     System.out.println(a+d); System.out.println(b);
8     System.out.println(c); System.out.println(d);
9   }
10 }
\end{verbatim}
Given an initial environment $\sigma_0$, 
derive the type binding environments for the method at each
use of an identifier and indicate where type lookups will occur.
\marks{30}

\begin{modelanswer}
0  $\sigma_0$ is starting environment\\
2  $\sigma_1 = \sigma_0 + \{a\rightarrow string,c\rightarrow int\}$\\
3  $\sigma_2 = \sigma_1 + \{b\rightarrow int,c\rightarrow String\}$ (overrides instance c)\\
4  lookup id c  in $\sigma_2$\\
5  $\sigma_3 = \sigma_2 + \{d\rightarrow int\}$ \\
6  lookup id b, then $\sigma_4 = \sigma_3 + \{a\rightarrow int\}$ (overrides instance a)\\
7  lookup a, d, b  in $\sigma_4$\\
8  lookup c, b in $\sigma_4$\\
9  discard $\sigma_4$ revert to $\sigma_1$\\
10 discard $\sigma_1$ revert to $\sigma_0$


3 marks per line.
\end{modelanswer}


\end{subquestions}

\newpage

\question

\begin{subquestions}

\subquestion
\begin{subsubquestions}
\subsubquestion
State two reasons why many 
programming language implementations require a memory model that 
implements a runtime stack?
\marks{10}
\subsubquestion
Explain in detail how a stack frame is pushed to the stack, and 
removed from the stack, during program execution. 
\marks{30}
\end{subsubquestions}

\begin{modelanswer}
Procedure or method calls, recursion, need for separate storage space for
parameters and locals.
Code generated for a proc/func does the pushing/popping.
\begin{verbatim}
caller g(...) calls callee f(a1,...,an)
calling code in g puts arguments to f at end of g frame
stores return address, old FP in control link
referenced through SP, incrementing SP
on entry to f, SP points to first argument g passes to f
old SP becomes current frame pointer FP
f then allocates frame by setting SP=(SP - framesize)
old SP becomes current frame pointer FP
f then initialises locals
on exit from f : SP = FP, removing frame
jumps to return address, restores old FP
\end{verbatim}
5 marks each reason
for answer to first part and explanation. 
25 marks for details.
\end{modelanswer}

\subquestion

\begin{subsubquestions}
\subsubquestion
Some programming language implementations avoid in some circumstances the
need to pass parameters via a stack frame. Outline what these circumstances
might be and why passing via the stack frame might be avoided.
\marks{20}

\begin{modelanswer}
Appropriate when leaf procs, interproc
reg alloc, dead variables, reg windows (but\ldots).
\end{modelanswer}

\subsubquestion
Explain three situations where the use of a 
stack frame to pass parameters cannot usually be avoided.
\marks{15}
\end{subsubquestions}
\begin{modelanswer}
Reg saves: when address is taken,
when call-by-ref,
when accessed by inner nesting,
value too big,
an array,
convention of save for partic reg prior to call,
spilling in exp evaluation,
saving a reg window.
\end{modelanswer}



\subquestion
Suppose  that a compiler translates a MiniJava-like language
to an intermediate representation (for example IR trees) 
that will include 
the calculations required to address variables in stack frames.
Draw or write down
the intermediate representation
required to access a local variable declared in a method.
Explain your answer.
\marks{25}

\begin{modelanswer}
MEM(BINOP(PLUS,TEMP fp, CONST k)) where k is offset of var in frame, fp the
register holding the framepointer. Has to compute place in frame.
\end{modelanswer}

\end{subquestions}

\end{questions}

\end{document}


