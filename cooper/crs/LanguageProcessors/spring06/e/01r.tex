%\documentstyle[cityexam,11pt]{article}
\documentclass[11pt]{cityexam}
\usepackage{semantic}
%\withmodelanswers
\begin{document}
\examnumber{905.57}
\begin{preamble}
\degrees{B.Eng. Degree in Computing \\ B.Eng. Degree in Software Engineering}
\part{II}
\title{Language Processors}
\examdate{5th September 2001}
\examtime{1400--1530}
\rubric{Answer {\sl TWO} questions \\ All questions carry equal marks}
\end{preamble}
\externals{P. Hall \\ M. Moulding}
\internals{D. Bolton}

\reservestyle{\nonterm}{\textit}
\nonterm{dec,decs,vardec,fundec,id,exp,tyfields,type-id,fundecs,expseq,op}
\reservestyle{\keyword}{\textbf}
\keyword{var,function,let,in,end,if,then,else,while,do,for,break,to,switch,of,case,default}


\begin{questions}


\question

\begin{subquestions}


\subquestion
Give a definition for regular expressions which shows how
arbitrary regular expressions can be constructed for a
given alphabet (set) of symbols. Do not include shorthand
notations you may know.
\marks{10}

\subquestion
Write down regular expressions that define a pattern to recognise
unsigned real numbers that are composed of an non-zero number of digits
followed by a decimal point (period) followed by an non-zero number of
digits. If there is more than a single digit in the string of 
digits before the point, the string must not begin with a `0'. 
If there is more than a single digit in the string of digits after the point, 
the string must not end with a `0'.
\marks{20}
\begin{modelanswer}
\begin{verbatim}
(([1-9][0-9]+)|[0-9]).([0-9]|[0-9]+[1-9])
\end{verbatim}
\end{modelanswer}

\subquestion
What is an ambiguous grammar? 
By giving suitable examples show that the
        following grammar is ambiguous. 
\begin{tabbing}
stmtxxx\=$\rightarrow$xxx\=if\kill
\it
exp \> $\rightarrow$ \> {\it exp\/} \verb!+! {\it exp} \\
\it
     \> $\mid$ \> {\it exp\/} \verb!-! {\it exp\/} \\
     \> $\mid$ \> {\it id}
\end{tabbing}
Here {\it id\/} stands for any identifier.
Explain the term {\em shift-reduce conflict\/} using as
an example the program fragment \verb!a+b-c!.
\marks{35}

\begin{modelanswer}
Two parse trees for a suitable expression (20).

Point at which a parser can either shift, or reduce by a rule. 
Here, reach the -, can shift, or reduce using first production (15).
\end{modelanswer}


\subquestion
Consider the following Tiger function:
\begin{verbatim}
1 function f(a:string, b:int, c:int)=
2       (print_int(b+c);
3        let var c := "hi"
4            var a := b
5            var b := "hello"
6        in print(b); print_int(a) 
7        end;
8        print_int(c); print_int(b);
9       )
\end{verbatim}
Given an initial environment $\sigma_0 = \{\textrm{a} \rightarrow \textrm{int}, \textrm{b} \rightarrow \textrm{string}\}$, derive the type binding environments for the function at each
use of an identifier and indicate where type lookups will occur.
\marks{35}

\begin{modelanswer}
0 $\sigma_0$ is starting environment\\
1 $\sigma_1 = \sigma_0 + \{a\rightarrow string,b\rightarrow int,c\rightarrow int\}$\\
2 lookup ids b, c  in $\sigma_1$\\
3 $\sigma_2 = \sigma_1 + \{c\rightarrow string\}$ (overrides arg c)\\
4 lookup id b, then $\sigma_3 = \sigma_2 + \{a\rightarrow int\}$ (overrides arg a)\\
5 $\sigma_4 = \sigma_3 + \{b\rightarrow string\}$ (overrides arg b\\
6 lookup id b, then a  in $\sigma_4$\\
6 discard $\sigma_4$, $\sigma_3$, $\sigma_2$ revert to $\sigma_1$\\
7 lookup c, b in $\sigma_1$\\
8 discard $\sigma_1$ revert to $\sigma_0$
\end{modelanswer}


\end{subquestions}

\newpage

\question

The reference manual for a Tiger'01-like programming language contains
the following definition for a kind of expression: 

The for-expression
\[
\<for> \; \<id> \; = \; \<exp>_1 \; \<to> \; \<exp>_2 \; \<do> \; \<exp>_3
\] 
iterates $\<exp>_3$ over each integer value of \<id> between
$\<exp>_1$ and $\<exp>_2$.
The variable \<id> is a new variable implicitly declared by the
\<for> statement, whose
scope covers only $\<exp>_3$,
and may not be assigned to. 
The upper and lower bounds $\<exp>_1$ and $\<exp>_2$
are evaluated only once,
prior to entering the body of the loop $\<exp>_3$.
If the upper bound is less than the lower, the
body is not executed.


\begin{subquestions}
\subquestion
Write down a BNF concrete syntax for the for-expression. 
\marks{10}

\begin{modelanswer}
Terminals in CAPS (trivial).
\begin{verbatim}
for-exp -> FOR id = exp TO exp DO exp
\end{verbatim}
\end{modelanswer}

\subquestion
Sketch a possible abstract syntax for the for-expression.
\marks{10}

\begin{modelanswer}
\begin{verbatim}
ForExp id (Exp lowerbound upperbound body) 
(be flexible about how this is expressed - it might be Java)
\end{verbatim}
\end{modelanswer}

\subquestion
Show how semantic actions in a grammar for a parser-generator such as CUP
can be used to produce abstract syntax trees for the for-expression.
\marks{30}

\begin{modelanswer}
\begin{verbatim}
        |   FOR:f ID:i ASSIGN exp:e1 TO exp:e2 DO exp:e3
            {: RESULT = new Absyn.ForExp(fleft, 
                  new Absyn.VarDec(ileft,sym(i),
                          inttype,e1),e2,e3); :}
\end{verbatim}
\end{modelanswer}

\subquestion
Explain how the scope rules for the implicitly declared variable might
be implemented.
\marks{15}

\begin{modelanswer}
By pushing a new scope for the for-loop itself. More marks if consideration
given to not assigning to it.
\end{modelanswer}

\subquestion
        Outline the intermediate representation
        that might be generated from your
	asbtract syntax trees in translation of
        a for-expression.
	You may wish to use a simple example to explain your 
	translation, eg:
\begin{verbatim}
for i = j to j+10 do ( x := i*i; print(x) )
\end{verbatim}
\marks{35}

\begin{modelanswer}
\begin{verbatim}
eval exp1 (here, j) into R1 (some TEMP loc ie a virtual reg)
eval exp2 (here, j+10) into R2 (some TEMP ie a reg)
MOVE(MEM(i),R1)
LABEL(Lstart)
MOVE(MEM(R1,MEM(i,16))
CJUMP(<,R2,R1,Lend,Lbody)
LABEL(Lbody)
code for body (here square and print)
MOVE(MEM(i),BINOP(+,MEM(i),1))
JUMP(Lstart)
LABEL(Lend)
\end{verbatim}
\end{modelanswer}

\end{subquestions}

\question


\begin{subquestions}

\subquestion
Choose a programming language you know well and describe
its run time storage requirements and functioning.
Clearly associate any storage structures you
mention with the implementation of particular 
language features.
\marks{30}

\begin{modelanswer}
Language features: procedure/method call and parameter passing,
lexical scoping, dynamic storage allocation and deallocation,
including object creation. Name stack, heap, why different.
More marks for good exposition of example language
(eg Java object creation, method call). Operation of
stack and heap (with mention of garbage collection).
\end{modelanswer}


\subquestion
What is a stack frame?
Outline a typical layout for a stack frame and describe 
each element of a frame and how it is pushed to the stack during
program execution. 
Explain the term {\em static link\/} and explain why a C or Java
implementation does not include static links in its stack frames.
%Comment on how local variables, arguments and non-local variables
%are addressed by the code generated for a procedure or method. 
\marks{40}

\begin{modelanswer}
A structure to store info local to a proc invocation.
Args, static link, locals, return address, regs/temp space.
More marks for mentioning code genned for a proc/func doing the
pushing.
%Locals, args: offsets from SP or FP; non-locals: follow access links
%(more for description of access link).
\end{modelanswer}

\subquestion
Explain why registers might be used for parameter passing and
suggest situations where passing in registers is particularly
appropriate. 
Outline situations where it is necessary for the code generated for a 
procedure or method to write registers to the stack.
\marks{30}

\begin{modelanswer}
Efficiency; particularly appropriate when leaf procs, interproc
reg alloc, dead variables, reg windows (but\ldots).
Reg saves: when address is taken,
when call-by-ref,
when accessed by inner nesting,
value too big,
an array,
convention of save for partic reg prior to call,
spilling in exp evaluation,
saving a reg window.
\end{modelanswer}

\end{subquestions}


\end{questions}

\end{document}

