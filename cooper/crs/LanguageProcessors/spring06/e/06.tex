%\documentstyle[cityexam,11pt]{article}
\documentclass[11pt]{bareexam}
\usepackage{semantic}
%\withmodelanswers
\begin{document}
%\examnumber{???}
%\begin{preamble}
%\degrees{???}

%\part{II}
%\title{Language Processors}
%\examdate{???}
%\examtime{???}
%\rubric{Answer {\sl TWO} questions \\ All questions carry equal marks}
%\end{preamble}
%\externals{???}
\internals{\ }

\reservestyle{\nonterm}{\textit}
\nonterm{Statement,Exp,dec,decs,vardec,fundec,id,exp,tyfields,type-id,fundecs,expseq,op}
\reservestyle{\keyword}{\textbf}
\keyword{var,function,let,in,end,if,then,else,while,do,for,break,to,switch,of,case,default}


\begin{questions}

\question

The reference manual for a MiniJava-like programming language contains
the following grammar rule for a do-while statement: 

\[
\<Statement> \; -> \; \<do> \; \<Statement> \; \<while> \; ( \<Exp>  ) \; ; 
\]

\begin{subquestions}
\subquestion
Write down or draw a possible abstract syntax for the do-while statement.
\marks{10}

\begin{modelanswer}
\begin{verbatim}
Do Statement Exp 
(be flexible about how this is expressed - it might be Java and have types;
it might be a picture)
\end{verbatim}
\end{modelanswer}

\subquestion
Show how semantic actions in a grammar for a parser-generator such as JavaCC
can be used to produce abstract syntax trees for the do-while statement.
\marks{25}

\begin{modelanswer}
\begin{verbatim}
Statement DoWhileStatement() :
{ Exp e;
  Statement s;
}
{
  "do" s=Statement() "while" "(" e=Expression() ")"
  { return new DoWhile(s,e); }
}
\end{verbatim}
5 marks for framework, 10 marks for syntax, 10 marks for correct actions.
\end{modelanswer}

\subquestion
Informally describe an appropriate typecheck for the do-while statement.
\marks{10}

\begin{modelanswer}
e must be a boolean expression.
\end{modelanswer}

\subquestion

Suppose a compiler for a MiniJava-like language 
translates all statements and expressions
into intermediate code (eg intermediate representation (IR) trees).

\begin{subsubquestions}

\subsubquestion
Draw or write down
the intermediate representation
required to access a local variable declared in a method.
Explain your answer.
\marks{20}

\begin{modelanswer}
MEM(BINOP(PLUS,TEMP fp, CONST k)) where k is offset of var in frame, fp the
register holding the framepointer. Has to compute place in frame.
\end{modelanswer}

\subsubquestion
Suppose the MiniJava-like language includes a do-while statement.
Outline the intermediate code that might be generated
in translation of the do-while statement. You may wish to use a simple
example to explain your translation, eg:
\begin{verbatim}
do { 
  x = i*i; System.out.println (x); i = i+1;
} while (i < j) ;
\end{verbatim}
You can assume that the expression tree for any variable \verb"v" is
simply \verb"TEMP v". You need not show translations for the body of the
example do-while statement (in braces in this example \verb+{...}+).
\marks{35}

\begin{modelanswer}
\begin{verbatim}
LABEL(Lstart)
code for body (here square, print and add 1)
CJUMP(<,TEMP i,TEMP j,Lstart,Lend)
LABEL(Lend)
\end{verbatim}
Might be expressed as trees.
\end{modelanswer}

\end{subsubquestions}
\end{subquestions}

\newpage

\question

\begin{subquestions}

\subquestion
The following regular expression recognises certain strings over the
alphabet $\{a,b,c\}$
\[
(a|c)((bc)|c)\!*c*
\]
Indicate which of these five strings are recognised by the above regular expression:
\begin{quote}
$cbc$, $abbc$, $c$, $abcbcbccc$, $ccbbcbcc$
\end{quote}
Also, show three more strings that are recognised by the above expression.
Finally, show two more strings consisting of 
the letters $a$, $b$ and $c$ that are \emph{not} 
recognised by the above regular expression.
\marks{30}

\begin{modelanswer}
Yes, No, Yes, Yes, No. 4 marks each. 
Five further strings, 2 marks each.
\end{modelanswer}

%\subquestion

\typeout{USE THIS IN RESIT!!!!!!}
%Using the character class extension
%for regular expressions, the pattern
%\verb![0-9]*! recognises decimal integers (and the empty string). 
%Use regular expressions to specify a pattern that recognises {\em only\/} 
%decimal integers that are divisible by 2. You may use the character 
%class extension if you wish.
%\marks{25}
%
%\begin{modelanswer}
%\begin{verbatim}
%[0-9]*[24680]
%\end{verbatim}
%\end{modelanswer}

\subquestion
Consider the following grammar for 
strings of balanced parentheses:
\begin{quote}
\begin{tabbing}
stmtxxx\=$\rightarrow$xxx\=if\kill
\it
S \> $\rightarrow$ \> {\it S\/} {\it S} \\
\it
S \> $\rightarrow$ \> \verb!(! {\it S\/} \verb!)! \\
\it
S \> $\rightarrow$ \> \verb!(! \verb!)!
\end{tabbing}
\end{quote}
Explain what it means for a context-free grammar to be ambiguous. 
Using your explanation, show that the balanced parentheses grammar is 
ambiguous using the shortest string that will illustrate the ambiguity.
\marks{30}

\begin{modelanswer}
Two parse trees, two derivations, for same sentence. 10 marks.
20 for the two trees of () () ().
\end{modelanswer}


\subquestion

Explain why left-recursion must be eliminated from
grammar productions which are to be used in
construction of a recursive-descent parser.
Write down a general rule for rewriting left-recursive
grammar productions to be right-recursive and use it to 
rewrite the following productions for an
arithmetic expression grammar to be right-recursive:
\begin{quote}
\begin{tabbing}
stmtxxx\=$\rightarrow$xxx\=if\kill
\it
E \> $\rightarrow$ \> {\it E\/} \verb!+! {\it T} \\
\it
E \> $\rightarrow$ \> {\it T\/} \\
\it
T \> $\rightarrow$ \> {\it T\/} \verb!*! {\it F} \\
\it
T \> $\rightarrow$ \> {\it F\/} \\
\it
F \> $\rightarrow$ \> ( {\it E\/} ) \\
\it
F \> $\rightarrow$ \> \verb!integer!
\end{tabbing}
\end{quote}
\vspace{-1em}
\marks{40}

\begin{modelanswer}
Because the usual way of writing the 
procedures leads to immediate recursive call. 5 marks.

Rule (10 marks):
\begin{verbatim}
A  -> A a | b
(a and b strings of terms and non-terms)
rewrites to 
A  -> b A'
A' -> a A' | empty
\end{verbatim}
Answer (30 marks):
\begin{verbatim}
E  -> T E'
E' -> + T E' | empty
T  -> F T'
T' -> * F T' | empty
F  -> ( E )  | integer
\end{verbatim}
\end{modelanswer}


\end{subquestions}

\newpage

\question

\begin{subquestions}
\subquestion
\begin{subsubquestions}
\subsubquestion
State two reasons why many 
programming language implementations require a memory model that 
implements a runtime stack?
\marks{10}
\subsubquestion
Explain in detail how a stack frame is pushed to the stack, and 
removed from the stack, during program execution. 
\marks{25}
\end{subsubquestions}

\begin{modelanswer}
Procedure or method calls, recursion, need for separate storage space for
parameters and locals.
Code generated for a proc/func does the pushing/popping.
\begin{verbatim}
caller g(...) calls callee f(a1,...,an)
calling code in g puts arguments to f at end of g frame
stores return address, old FP in control link
referenced through SP, incrementing SP
on entry to f, SP points to first argument g passes to f
old SP becomes current frame pointer FP
f then allocates frame by setting SP=(SP - framesize)
old SP becomes current frame pointer FP
f then initialises locals
on exit from f : SP = FP, removing frame
jumps to return address, restores old FP
\end{verbatim}
5 marks each reason
for answer to first part and explanation. 
25 marks for details.
\end{modelanswer}

\subquestion
\begin{subsubquestions}
\subsubquestion
Some programming language implementations avoid in some circumstances the
need to pass parameters via a stack frame. Outline what these circumstances
might be and why passing via the stack frame might be avoided.
\marks{15}

\begin{modelanswer}
Appropriate when leaf procs, interproc
reg alloc, dead variables, reg windows (but\ldots).
\end{modelanswer}

\subsubquestion
Explain three situations where the use of a 
stack frame to pass parameters cannot usually be avoided.
\marks{15}
\end{subsubquestions}
\begin{modelanswer}
Reg saves: when address is taken,
when call-by-ref,
when accessed by inner nesting,
value too big,
an array,
convention of save for partic reg prior to call,
spilling in exp evaluation,
saving a reg window.
\end{modelanswer}

\subquestion
\begin{subsubquestions}
\subsubquestion
Explain the difference between \emph{caller-save\/} and \emph{callee-save\/}
registers. 
\marks{10}
\subsubquestion
Study the following methods and suggest for each whether a caller-save 
or callee-save register is appropriate for variable \verb+x+. 
Explain your answers.
\begin{verbatim}
int f (int a) { int x; x=a+1; g(x); return x+2; }

void p (int y) { int x; x=y+q(y); q(2); q(y+1)}
\end{verbatim}
\marks{25}
\end{subsubquestions}

\begin{modelanswer}
Caller-save if code for caller of a func saves and restores the reg value
around a func call. Callee save if code for a func does it. First method x in
callee-save, since x live across the method calls. Second caller-save, since
x not live after x=y+q(y) (so code generated shouldn't save it).

10 marks for explanation,  12.5 each for x answers and explanations.
\end{modelanswer}

\end{subquestions}

\end{questions}

\end{document}


