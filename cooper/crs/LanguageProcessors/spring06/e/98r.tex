\documentstyle[/homes/daveb/doc/exam_style/cityexam,11pt]{article}
\begin{document}
\withmodelanswers
\examnumber{904.02}
\begin{preamble}
\degrees{BSc Degree in Computer Science \\  B.Eng Degree in Software Engineering}
\part{II RESIT}
\title{Language Processors (348)}
\examdate{??? 1998}
\examtime{9.30--11.00}
\rubric{Answer two questions \\ All questions carry equal marks}
\end{preamble}
\externals{P Lee \\ J Woodcock}
\internals{DJ Bolton}

\begin{questions}

% Herewith four questions for Language Processors

\newenvironment{bnf}{
    \begin{tabbing}
    XXXXXX \= XXXX \= XXXXXXXXXXXXXXXXXXXXXXXXXXXXXXXXXX\kill}{
    \end{tabbing}
    }

\newcommand{\nt}[1]{{\it #1\/}}
\newcommand{\Or}[1]{\ \ $|$ \ \ #1}
\newcommand{\Rule}[2]{{\it #1} \> $\rightarrow$ \> #2}

\question

\begin{subquestions}

\subquestion

Give a definition for regular expressions which shows how
arbitrary regular expressions can be constructed for a
given alphabet (set) of symbols. Do not include shorthand
notations you may know.
\marks{10}

\subquestion
        In some editors, pattern matching notation is
        provided as detailed in the table below.
        For example, the expression \verb"{.*}" 
        matches any sequence of characters with enclosing
        curly brackets.
        Show how the patterns in the table can be represented by regular
        expressions. Show also that regular expression notation is
        more powerful than the editor pattern notation by
        finding examples of patterns that can be expressed in the
        regular expression notation but not in the editor pattern
        notation.

%\vspace{0.2in}

\begin{center}
\begin{tabular}{|c|c|c|}        \hline
Expression & Matches & Examples \\ \hline
\verb"c"                & single character \verb"c"      & \verb"c", \verb"x", \verb"!"\\                     
\verb"*"                & zero or more characters        & \verb"lo*p" matches \verb"lp", \verb"lop", \verb"loop", \verb"looop"\ldots\\
\verb"."                & any character                  & \verb"c.t" matches \verb"cat", \verb"cct", \verb"c%t"\ldots\\
\verb"["{\it s\/}\verb"]" & any character in {\it s}     & \verb"c[aou]t"  matches \verb"cat", \verb"cot", \verb"cut"\\
 \hline
\end{tabular}
\end{center}
\marks{30}

\begin{modelanswer}
\begin{verbatim}
c -- c   3\\
* -- a*, a in alphabet    5\\
. -- a1 or a2 \ldots for all a in alphabet    5 \\
\[s\] -- a1 or a2 \ldots for all a in s    7

* in editor constrained to work on characters; alternates in
\[..\] may only be characters (15).
\end{verbatim}
\end{modelanswer}  


\subquestion

Explain what is meant by the argument passing mechanisms usually
        known as
                \begin{enumerate}
                \item call-by-value
                \item call-by-reference
                %\item call-by-value-result
                \item call-by-name
                \end{enumerate}
        Deduce for each mechanism what is the effect of
        the following procedure:
\begin{verbatim}
       PROCEDURE pass (p,q,r: INTEGER);
       BEGIN
            q := q + 1;
            r := r + p
       END pass;
\end{verbatim}
        when the sequence of assignments and a procedure call
        shown below is executed, assuming that all arguments are
        passed with the same mechanism, and that \verb"a" and
        \verb"b" are variables of type INTEGER.
        Write down what is printed in each case by the \verb"WriteInt"
        call, and clearly justify your answer by reference to your
        description of the argument passing mechanism.
\begin{verbatim}
       a := 2;
       b := 3;
       pass (a+b,a,a);
       InOut.WriteInt (a)
\end{verbatim}
Although the program is written in a Modula-2-like syntax,
the Modula-2 parameter-passing semantics do not apply and you must
instead consider each of the mechanisms mentioned.
\marks{60}

\begin{modelanswer}
\begin{verbatim}
cbv explanation -- answer is 2.
cbr explanation -- 
   a := a+1 = 3
   a := a+5 = 8    answer 8
[ cbvr explanation NOT ASKED HERE
   5,2,2 passed
   q := q+1 = 3
   r := r+5 = 7    answer 7 or 3 depending on copy back order ]
cbn explanation --
   a+b,a,a passed
   a := a + 1 = 3
   a := a + a + b = 3 + 3 + 3 = 9 
\end{verbatim}
\end{modelanswer}

\end{subquestions}

%\newpage

\question

\begin{subquestions}

\subquestion
The following is an extract from a possible BNF syntax for
an imperative programming language: 
\begin{bnf}
\Rule{\nt{Compound\ }}{\nt{Instruction} \verb";" \nt{Compound}\Or{\nt{Instruction}}} \\
\Rule{\nt{Instruction\ }}{\nt{Cond}\Or{\ldots}} \\
\Rule{\nt{Cond}}{\verb"if" \nt{BoolExp} \verb"then" \nt{Compound$_1$} \verb"else" \nt{Compound$_2$} \verb"end"}\\
\Rule{\nt{IntExp}}{\nt{IntVar}\Or{\nt{IntConstant}}\Or{\nt{IntExp} \verb"+" \nt{IntExp}}
        \Or{\nt{IntExp} \verb"*" \nt{IntExp}}}\\
\Rule{\nt{BoolExp}}{\nt{IntExp} \verb"=" \nt{IntExp}\Or{\nt{IntExp} \verb"<=" \nt{IntExp}}}
\end{bnf}
where \nt{IntVar} is a variable of type INTEGER, \nt{IntConstant}
is an INTEGER type constant (eg 3).

The meaning of a \nt{Cond} is as follows:
\begin{itemize}
\item On entry to the instruction, the \nt{BoolExp} is 
        evaluated. 
\item If the value of the \nt{BoolExp} is true, the effect of
        the \nt{Cond} is the effect of \nt{Compound$_1$}.
\item Otherwise, when the \nt{BoolExp} evaluates to false,
        the effect of the \nt{Cond} is the effect of
        of \nt{Compound$_2$}.
\end{itemize}

\begin{subsubquestions}
%\subquestion
%Show that the grammar rules for \nt{IntExp} are ambiguous.
%\marks{20}
\subsubquestion
Write down an abstract syntax for \nt{Cond}, 
\nt{BoolExp} and \nt{IntExp}.
\marks{20}
\begin{modelanswer}
\begin{verbatim}
Cond -> COND BoolExp Compound Compound
BoolExp     -> BEQ IntExp IntExp | BLE IntExp IntExp
IntExp      -> INTVAR name | INTCONST int | INTPLUS IntExp IntExp |
                 INTMULT IntExp IntExp
\end{verbatim}
\end{modelanswer}
\subsubquestion
        Outline the structure of the
        code that might be generated in translation of
        a \nt{Cond} with an example 
        \nt{BoolExp} (\verb"a+1 <= b*2", say). 
        Assume code generation for a simple
        stack machine, and briefly explain the
        instructions you use in your outline.
\marks{40}

\begin{modelanswer}
\begin{verbatim}
code for RHS IntExp, result on stack top
code for LHS IntExp, result on stack top
LE                   top <= top-1
JUMPFALSE LABEL1
code for Compound1
JUMP LABEL2
LABEL1:
code for Compound2
LABEL2:
\end{verbatim}
Plus explanation.
\end{modelanswer}

\subsubquestion
        Construct a syntax-directed translation for 
        \nt{Cond} and \nt{BoolExp} which produces such code.
        Describe any attributes you
        introduce, and make clear any assumptions you make about 
        the attributes and code generated for the
        expressions \nt{IntExp}, but you need not produce translations for
        \nt{IntExp}.
\marks{40}
\begin{modelanswer}
\begin{verbatim}
Cond -> COND BoolExp { 
                         elsepart := newlabel()
                         gen (JUMPFALSE elsepart) }
               Compound {
                         end := newlabel()
                         gen (JUMP end)
                         gen (LABEL elsepart) }
               Compound {
                         gen (LABEL end) }
BoolExp     -> BEQ IntExp IntExp { gen (EQ) }
            |  BLE IntExp IntExp { gen (LE) }
\end{verbatim}
Plus explanation for attrs, results of IntExp translation.
\end{modelanswer}
\end{subsubquestions}

\end{subquestions}

\question

\begin{subquestions}


\subquestion
%Briefly describe the function of each
%of the subdivisions (areas), and
Discuss the programming language features which influence
the organisation and functioning of run time storage,
clearly associating any storage structures you
mention with the implementation of particular language features.
Choose a language you know well and briefly describe
its run time storage requirements and functioning.
%Explain why a typical C language memory model does not
%have an area set aside for dynamically allocated
%storage.
\marks{40}
\begin{modelanswer}
Language features: procedure call and parameter passing,
lexical scoping, dynamic storage allocation and deallocation,
including object creation. Name stack, heap, why different.
More marks for example languages
(eg Eiffel object creation, Pascal `new' etc). Operation of
stack and heap (with mention of garbage collection).
Particular language example gets more marks.
\end{modelanswer}

\subquestion

Comment on how local variables,
parameters (arguments) and non-local variables
are addressed by the code generated for a
procedure when a run-time stack is in use,
and explain the need for a register `spill' area
in some records.
\marks{25}

\subquestion
Explain what is 
meant by {\em fragmentation}, and mention some methods
for reducing fragmention in heaps.  
{\em Garbage\/} and {\em dangling pointers}
can occur in the implementation of programming languages which
support explicit deallocation of heap objects. Define
these terms and explain using simple example program
fragments how they arise.
%Sketch the simple mark-sweep garbage collection algorithm
%for automatic deallocation of inaccessible heap objects.
\marks{35}

\end{subquestions}

\end{questions} 

\end{document}

