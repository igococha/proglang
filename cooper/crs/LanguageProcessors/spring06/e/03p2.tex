%\documentstyle[cityexam,11pt]{article}
\documentclass[11pt]{cityexam}
\usepackage{semantic}
%\withmodelanswers
\begin{document}
\examnumber{509.xxx}
\begin{preamble}
\degrees{B.Eng. Computing \\ Professional Pathway  3 - Computing \\ Professional Pathway  3 - Computing (DISC)}

\part{II}
\title{Language Processors}
\examdate{May 9th 2003}
\examtime{9.00-10.30}
\rubric{Answer {\sl TWO} questions \\ All questions carry equal marks}
\end{preamble}
\externals{Professor M.E.C\@. Hull\\ Professor M\@. Moulding}
\internals{D. Bolton}

\reservestyle{\nonterm}{\textit}
\nonterm{dec,decs,vardec,fundec,id,exp,tyfields,type-id,fundecs,expseq,op}
\reservestyle{\keyword}{\textbf}
\keyword{var,function,let,in,end,if,then,else,while,do,for,break,to,switch,of,case,default}


\begin{questions}


\question

\begin{subquestions}


\subquestion
The following regular expression recognises certain strings consisting of the
letters $a$, $b$ and $c$:
\[
a((ab)|(ac))*c
\]
Indicate which of these five strings are recognised by the above regular expression:
\begin{quote}
$aacc$, $abac$, $ac$, $ababababacac$, $aabacc$
\end{quote}
Also, show three more strings that are recognised by the above expression.
Finally, show two more strings consisting of 
the letters $a$, $b$ and $c$ that are \emph{not} 
recognised by the above regular expression.
\marks{30}

\begin{modelanswer}
Yes, No, Yes, No, Yes. 3 marks each. 
Five further strings, 3 marks each.
\end{modelanswer}


\subquestion
Most programming languages allow the same symbol to denote both the 
subtraction operator (in $x-y$) and the unary negation operator (in $-x$).
Make clear the difficulty this causes for a parser-generator (for example, CUP)
and explain how it may be overcome.
\marks{20}

\begin{modelanswer}
Usually the precedence of the operator needs to be different in the two different
contexts, eg highest for unary minus, and equal to plus etc for binary minus.
Can be overcome by assigning a specific precedence to the rule for the unary case.
So a 'pseudo-terminal' is introduced of appropriate precedence and the
appropriate rule is annotated with it: 
\begin{verbatim}
%left + - 
%left * /
%left UM

...
exp : exp + exp
    | ...
    | - exp %prec UM
\end{verbatim}
\end{modelanswer}

\subquestion
Explain what it means for a context-free grammar to be ambiguous. 
Write down an ambiguous grammar and show why it is ambiguous.
\marks{20}

\begin{modelanswer}
Two parse trees, two derivations, for same sentence. 5 marks.
5 marks for a grammar. 10 for the two trees/derivations.
\end{modelanswer}

\subquestion
Consider the following Tiger function:
\begin{verbatim}
1 function f(a:string, b:int, c:int)=
2       (print_int(b+c);
3        let var c := "hi"
4            var a := b
5            var b := "hello"
6        in print(b); print_int(a) 
7        end;
8        print_int(c); print_int(b);
9       )
\end{verbatim}
Given an initial environment $\sigma_0 = \{\textrm{a} \rightarrow \textrm{int}, \textrm{b} \rightarrow \textrm{string}\}$, 
derive the type binding environments for the function at each
use of an identifier and indicate where type lookups will occur.
\marks{30}

\begin{modelanswer}
0 $\sigma_0$ is starting environment\\
1 $\sigma_1 = \sigma_0 + \{a\rightarrow string,b\rightarrow int,c\rightarrow int\}$\\
2 lookup ids b, c  in $\sigma_1$\\
3 $\sigma_2 = \sigma_1 + \{c\rightarrow string\}$ (overrides arg c)\\
4 lookup id b, then $\sigma_3 = \sigma_2 + \{a\rightarrow int\}$ (overrides arg a)\\
5 $\sigma_4 = \sigma_3 + \{b\rightarrow string\}$ (overrides arg b\\
6 lookup id b, then a  in $\sigma_4$\\
6 discard $\sigma_4$, $\sigma_3$, $\sigma_2$ revert to $\sigma_1$\\
7 lookup c, b in $\sigma_1$\\
8 discard $\sigma_1$ revert to $\sigma_0$
\end{modelanswer}



\end{subquestions}

\newpage

\question

The reference manual for a Tiger-like programming language contains
the following definition for a kind of expression: 

The if-expression
\[
\<if> \; \<exp>_1 \; \<then> \; \<exp>_2 \; \<else> \; \<exp>_3
\] 
evaluates the expression $\<exp>_1$. If the result is non-zero the if-expression
yields the result of evaluating $\<exp>_2$; otherwise it yields the
result of evaluating $\<exp>_3$.

\begin{subquestions}
\subquestion
Write down a BNF concrete syntax for the if-expression. 
\marks{10}

\begin{modelanswer}
Terminals in CAPS (trivial).
\begin{verbatim}
if-exp -> IF exp THEN exp ELSE exp
\end{verbatim}
\end{modelanswer}

\subquestion
Sketch a possible abstract syntax for the if-expression.
\marks{10}

\begin{modelanswer}
\begin{verbatim}
IfExp Exp Exp Exp
(be flexible about how this is expressed - it might be Java)
\end{verbatim}
\end{modelanswer}

\subquestion
Show how semantic actions in a grammar for a parser-generator such as CUP
can be used to produce abstract syntax trees for the if-expression. 
\marks{20}

\begin{modelanswer}
\begin{verbatim}
        |   IF:i exp:e1 THEN exp:e2 ELSE exp:e3
            {: RESULT = new Absyn.IfExp(ileft,e1,e2,e3); :}
\end{verbatim}
\end{modelanswer}

\subquestion
Informally describe an appropriate typecheck for the if-expression.
\marks{20}

\begin{modelanswer}
Exp e1 must be an integer, and e2 and e3 must be the same type 
(which is the type of the whole expression), or both must produce no value,
and then  the whole expression produces no value.
\end{modelanswer}

\subquestion

Suppose a Tiger compiler translates all expressions and subexpressions
into intermediate code (eg expression trees).
Outline the intermediate code that might be generated
in translation of the if-expression:
\begin{verbatim}
if a < b then c := a else c := b
\end{verbatim}
You can assume that the expression tree for any variable \verb"v" is
simply \verb"TEMP v".
\marks{40}

\begin{modelanswer}
\begin{verbatim}
SEQ(SEQ(CJUMP(LT, TEMP a, TEMP b, L0, L1),
        SEQ(SEQ(SEQ(LABEL L0,
                  MOVE(TEMP c, TEMP a)),
                JUMP(NAME L2)),
            SEQ(SEQ(LABEL L1,
                  MOVE(TEMP c, TEMP b)),
                JUMP(NAME L2)))),
     LABEL L2)

(might be a longer translation where a < b is evaluated to 
the 0 or 1 and then checked)
\end{verbatim}
\end{modelanswer}

\end{subquestions}

\question


\begin{subquestions}

\subquestion
Choose a programming language you know well and describe
how run-time storage is organised and managed during program 
execution.
Clearly associate any storage structures you
mention with the implementation of particular 
language features.
\marks{25}

\begin{modelanswer}
Language features: procedure/method call and parameter passing,
lexical scoping, dynamic storage allocation and deallocation,
including object creation. Name stack, heap, why different.
More marks for good exposition of example language
(eg Java object creation, method call). Operation of
stack and heap (with mention of garbage collection).
\end{modelanswer}


\subquestion
What is a stack frame?
Outline a typical layout for a stack frame and describe 
each element of a frame. 
%and how it is pushed to the stack during
%program execution. 
%Explain the term {\em static link\/} and explain why a C or Java
%implementation does not include static links in its stack frames.
Comment on how local variables, arguments and non-local variables
are addressed by the code generated for a procedure or method. 
\marks{25}

\begin{modelanswer}
A structure to store info local to a proc invocation.
Args, static link, locals, return address, regs/temp space.
More marks for mentioning code genned for a proc/func doing the
pushing.
Locals, args: offsets from SP or FP; non-locals: follow static links
(more for description of static link).
\end{modelanswer}

\subquestion
Explain why registers might be used for parameter passing and
suggest situations where passing in registers is particularly
appropriate. 
Outline situations where it is necessary for the code generated for a 
procedure or method to write registers to the stack.
\marks{30}

\begin{modelanswer}
Efficiency; particularly appropriate when leaf procs, interproc
reg alloc, dead variables, reg windows (but\ldots).
Reg saves: when address is taken,
when call-by-ref,
when accessed by inner nesting,
value too big,
an array,
convention of save for partic reg prior to call,
spilling in exp evaluation,
saving a reg window.
\end{modelanswer}

\subquestion
Explain the difference between \emph{caller-save\/} and \emph{callee-save\/}
registers. Why might caller-save registers sometimes not be saved?
\end{subquestions}
\marks{20}

\begin{modelanswer}
Caller-save if code for caller of a func saves and restores the reg value
around a func call. Callee save if code for a func does it. Might not be
saved when analysis shows that a value of some var will not be needed following
the call - put in caller-save reg but NOT save it ahead of call.
\end{modelanswer}

\end{questions}

\end{document}

