%\documentstyle[cityexam,11pt]{article}
\documentclass[11pt]{bareexam}
\usepackage{semantic}
%\withmodelanswers
\begin{document}
%\examnumber{???}
%\begin{preamble}
%\degrees{???}

%\part{II}
%\title{Language Processors}
%\examdate{???}
%\examtime{???}
%\rubric{Answer {\sl TWO} questions \\ All questions carry equal marks}
%\end{preamble}
%\externals{???}
\internals{\ }

\reservestyle{\nonterm}{\textit}
\nonterm{Statement,Exp,dec,decs,vardec,fundec,id,exp,tyfields,type-id,fundecs,expseq,op}
\reservestyle{\keyword}{\textbf}
\keyword{var,function,let,in,end,if,then,else,while,do,for,break,to,switch,of,case,default}


\begin{questions}

\question

\begin{subquestions}

\subquestion
The following regular expression recognises certain strings consisting of the
letters $a$, $b$ and $c$:
\[
a(b|(bc))\!*c*
\]
Indicate which of these five strings are recognised by the above regular expression:
\begin{quote}
$acc$, $abac$, $a$, $abcbcbccc$, $abbbccbc$
\end{quote}
Also, show three more strings that are recognised by the above expression.
Finally, show two more strings consisting of 
the letters $a$, $b$ and $c$ that are \emph{not} 
recognised by the above regular expression.
\marks{25}

\begin{modelanswer}
Yes, No, Yes, Yes, Yes. 3 marks each. 
Five further strings, 2 marks each.
\end{modelanswer}


\subquestion
Consider the following grammar, which we will call {\it E\/}:
\begin{quote}
\begin{tabbing}
stmtxxx\=$\rightarrow$xxx\=if\kill
\it
E \> $\rightarrow$ \> {\it E\/} \verb!+! {\it E} \\
\it
E \> $\rightarrow$ \> {\it E\/} \verb!-! {\it E} \\
\it
E \> $\rightarrow$ \> ( {\it E\/} ) \\
\it
E \> $\rightarrow$ \> \verb!num!
\end{tabbing}
\end{quote}

\begin{subsubquestions}
\subsubquestion
Explain what it means for a context-free grammar to be ambiguous. 
Show that grammar {\it E\/} is ambiguous.
\marks{20}

\begin{modelanswer}
Two parse trees, two derivations, for same sentence. 5 marks.
15 for the two trees/derivations.
\end{modelanswer}

\subsubquestion
Explain why grammar {\it E\/} is not suitable to form the basis for a 
recursive descent parser.
\marks{10}

\begin{modelanswer}
Because it's left-recursive and so the usual way of writing the 
procedures leads to immediate recursive call.
\end{modelanswer}

\subsubquestion
Rewrite the rules to obtain an equivalent grammar which can
be used as the basis for a recursive descent parser.
\marks{25}

\begin{modelanswer}
\begin{verbatim}
E -> num E'
E -> (E) E'
E' -> + E E'
E' -> - E E'
E' -> empty
\end{verbatim}
\end{modelanswer}

\end{subsubquestions}

\subquestion
Consider the following Java method:
\begin{verbatim}
1  class A {
2   String a; int c;
3   public void f(int b, String c) {
4     System.out.println(c);
5     int d = 3;
6     int a = b;
7     System.out.println(a+d); System.out.println(b);
8     System.out.println(c); System.out.println(d);
9   }
10 }
\end{verbatim}
Given an initial environment $\sigma_0$, 
derive the type binding environments for the method at each
use of an identifier and indicate where type lookups will occur.
\marks{20}

\begin{modelanswer}
0  $\sigma_0$ is starting environment\\
2  $\sigma_1 = \sigma_0 + \{a\rightarrow string,c\rightarrow int\}$\\
3  $\sigma_2 = \sigma_1 + \{b\rightarrow int,c\rightarrow String\}$ (overrides instance c)\\
4  lookup id c  in $\sigma_2$\\
5  $\sigma_3 = \sigma_2 + \{d\rightarrow int\}$ \\
6  lookup id b, then $\sigma_4 = \sigma_3 + \{a\rightarrow int\}$ (overrides instance a)\\
7  lookup a, d, b  in $\sigma_4$\\
8  lookup c, d in $\sigma_4$\\
9  discard $\sigma_4$ revert to $\sigma_1$\\
10 discard $\sigma_1$ revert to $\sigma_0$
\end{modelanswer}

2 marks for each line

\end{subquestions}

\newpage

\question

The reference manual for a MiniJava-like programming language contains
the following grammar for a for-statement: 

\[
\<Statement> \; -> \; \<for> \; ( \<id> = \<Exp> \; ; \; \<Exp> \; ; \; \<id> = 
\<Exp> ) \; \<Statement>
\]


\begin{subquestions}

\subquestion
Sketch a possible abstract syntax for the for-statement.
\marks{20}

\begin{modelanswer}
\begin{verbatim}
For Id Exp Exp Id Exp Statement
(be flexible about how this is expressed - it might be Java and have types)
\end{verbatim}
\end{modelanswer}

\subquestion
Show how semantic actions in a grammar for a parser-generator such as JavaCC
can be used to produce abstract syntax trees for the for-statement. 
\marks{30}

\begin{modelanswer}
\begin{verbatim}
Statement ForStatement() : 
{ Identifier i1, i2;
  Exp e1, e2, e3;
  Statement s;
}
{
  "for" "(" i1=Identifier() "=" e1=Expression() ";"
            e2=Expression() ";"
            i2=Identifier() "=" e3=Expression() 
         ")" s=Statement() 
  { return new For(i1,e1,e2,i2,e3,s); }
}
\end{verbatim}

10 marks for syntactic structure, 15 for actions, 5 for a small explanation.
\end{modelanswer}

\subquestion
Informally describe an appropriate typecheck for the for-statement.
\marks{10}

\begin{modelanswer}
i1 should be same type as e1, i2 same type as e3 (5 marks);
e2 must be a boolean expression (5 marks). 
\end{modelanswer}

\subquestion


Suppose a compiler for a MiniJava-like language that includes
a for-statement translates all statements and expressions
into intermediate code (eg intermediate representation (IR) trees).

%\begin{subsubquestions}
%
%\subsubquestion
%Draw or write down
%the intermediate representation
%required to access a local variable declared in a method.
%Explain your answer.
%\marks{15}
%
%\begin{modelanswer}
%MEM(BINOP(PLUS,TEMP fp, CONST k)) where k is offset of var in frame, fp the
%register holding the framepointer. Has to compute place in frame.
%\end{modelanswer}
%
%\subsubquestion
Outline the intermediate code that might be generated
in translation of the for-statement. You may wish to use a simple
example to explain your translation, eg:
\begin{verbatim}
for (i = j; i < k; i=i+1) 
   { x = i*i; System.out.println (x); }
\end{verbatim}
You can assume that the expression tree for any variable \verb"v" is
simply \verb"TEMP v". Do not show translations for the body of the
example for-statement (in braces in this example \verb+{...}+).
\marks{40}

\begin{modelanswer}
\begin{verbatim}
MOVE(TEMP i,TEMP j)
LABEL(Lstart)
CJUMP(<,TEMP i,TEMP k,Lbody,Lend)
LABEL(Lbody)
code for body (here square and print)
MOVE(TEMP i,BINOP(+,TEMP i,1))
JUMP(Lstart)
LABEL(Lend)
\end{verbatim}
Might be expressed as trees.
\end{modelanswer}

%\end{subsubquestions}

\end{subquestions}

\question

\begin{subquestions}
\subquestion
Why do many programming language implementations require a memory model that 
implements a runtime stack?
Explain in detail how a stack frame is pushed to the stack, and 
removed from the stack, during program execution. 
\marks{30}

\begin{modelanswer}
Procedure or method calls, recursion, need for separate storage space for
parameters and locals.
Code generated for a proc/func does the pushing/popping.
\begin{verbatim}
caller g(...) calls callee f(a1,...,an)
calling code in g puts arguments to f at end of g frame
stores return address
referenced through SP, incrementing SP
on entry to f, SP points to first argument g passes to f
old SP becomes current frame pointer FP
f then allocates frame by setting SP=(SP - framesize)
old SP becomes current frame pointer FP
f then initialises locals
on exit from f : SP = FP, removing frame
jumps to return address
\end{verbatim}
10 marks for answer to first part and explanation. 20 marks for details.
\end{modelanswer}

\subquestion
Some programming language implementations avoid in some circumstances the
need to pass parameters via a stack frame. Outline what these circumstances
might be and why passing via the stack frame might be avoided.
Also, outline situations where the use of a 
stack frame to pass parameters cannot usually be avoided.
\marks{25}

\begin{modelanswer}
Appropriate when leaf procs, interproc
reg alloc, dead variables, reg windows (but\ldots).

Reg saves: when address is taken,
when call-by-ref,
when accessed by inner nesting,
value too big,
an array,
convention of save for partic reg prior to call,
spilling in exp evaluation,
saving a reg window.

5 marks for explanation, 2 each for details.
\end{modelanswer}

\subquestion
Suppose  that a compiler translates a MiniJava-like language
to an intermediate representation (for example IR trees) 
that will include 
the calculations required to address variables in stack frames.
Draw or write down
the intermediate representation
required to access a local variable declared in a method.
Explain your answer.
\marks{20}

\begin{modelanswer}
MEM(BINOP(PLUS,TEMP fp, CONST k)) where k is offset of var in frame, fp the
register holding the framepointer. Has to compute place in frame.
\end{modelanswer}

\subquestion
Explain the difference between \emph{caller-save\/} and \emph{callee-save\/}
registers. Study the following methods and suggest for each whether a caller-save 
or callee-save register is appropriate for variable \verb+x+. Explain your answers.
\begin{verbatim}
int f (int a) { int x; x=a+1; g(x); return x+2; }

void p (int y) { int x; x=y+q(y); q(2); q(y+1)}
\end{verbatim}
\marks{25}

\begin{modelanswer}
Caller-save if code for caller of a func saves and restores the reg value
around a func call. Callee save if code for a func does it. First method x in
callee-save, since x live across the method calls. Second caller-save, since
x not live after x=y+1 (so code generated shouldn't save it).

5 marks for explanation, 10 each for x answers and explanations.
\end{modelanswer}

\end{subquestions}

\end{questions}

\end{document}


