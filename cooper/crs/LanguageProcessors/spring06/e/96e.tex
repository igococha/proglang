\documentstyle[/homes/daveb/doc/exam_style/cityexam,11pt]{article}
\begin{document}
\examnumber{00.00}
\begin{preamble}
\degrees{EC2 retread}
\part{II}
\title{Programming Systems and Principles of Programming Languages}
\examdate{XXX, 00 XXX, 1994}
\examtime{00 to 00}
\rubric{Two from section A, two from B}
\end{preamble}
\externals{XXX \\ XXX}
\internals{S Hunt \\ DJ Bolton}

\begin{questions}

% Herewith four questions for Language Processors

\newenvironment{bnf}{
    \begin{tabbing}
    XXXXXX \= XXXX \= XXXXXXXXXXXXXXXXXXXXXXXXXXXXXXXXXX\kill}{
    \end{tabbing}
    }

\newcommand{\nt}[1]{{\it #1\/}}
\newcommand{\Or}[1]{\ \ $|$ \ \ #1}
\newcommand{\Rule}[2]{{\it #1} \> $\rightarrow$ \> #2}

\question

\begin{subquestions}

\subquestion

Give a definition for regular expressions which shows how
arbitrary regular expressions can be constructed for a
given alphabet (set) of symbols. Do not include shorthand
notations you may know.
\marks{15}

\subquestion
        In some editors, pattern matching notation is
        provided as detailed in the table below.
        For example, the expression \verb"{.*}" 
        matches any sequence of characters with enclosing
        curly brackets.
        Show how these patterns can be represented by regular
        expressions. Show also that regular expression notation is
        more powerful than the editor pattern notation by
        finding examples of patterns that can be expressed in the
        regular expression notation but not in the editor pattern
        notation.

%\vspace{0.2in}

\begin{center}
\begin{tabular}{|c|c|c|}        \hline
Expression & Matches & Examples \\ \hline
\verb"c"                & single character \verb"c"      & \verb"c", \verb"x", \verb"!"\\                     
\verb"*"                & zero or more characters        & \verb"lo*p" matches \verb"lp", \verb"lop", \verb"loop", \verb"looop"\ldots\\
\verb"."                & any character                  & \verb"c.t" matches \verb"cat", \verb"cct", \verb"c%t"\ldots\\
\verb"["{\it s\/}\verb"]" & any character in {\it s}     & \verb"c[aou]t"  matches \verb"cat", \verb"cot", \verb"cut"\ldots\\
 \hline
\end{tabular}
\end{center}
\marks{30}

\vspace{0.1in}

\subquestion
        Explain why left-recursion must be eliminated from
        grammar productions which are to be used in
        construction of a recursive-descent parser.
        Write down a general rule for rewriting left-recursive
        grammar productions to be right-recursive, and use the rule
        to rewrite the following productions:
\begin{bnf}
\Rule{\nt{expr}}{\nt{expr} \verb"+" \nt{term}\Or{\nt{term}}}
\end{bnf}
\marks{30}

\subquestion
What is an ambiguous grammar? By giving suitable examples show that the
        following grammar is ambiguous. 
\begin{tabbing}
stmtxxx\=$\rightarrow$xxx\=if\kill
\it
stmt \> $\rightarrow$ \> \verb"if" {\it expr\/} \verb"then" {\it stmt} \\
\it
     \> $\mid$ \> \verb"if" {\it expr\/} \verb"then" {\it stmt\/} \verb"else" {\it stmt} \\
     \> $\mid$ \> {\it other}
\end{tabbing}
Here {\it other\/} stands for any other statement.
\marks{25}


\end{subquestions}


\question


The following is an extract from the BNF syntax for
a Modula-2-like programming language:
\begin{bnf}
\Rule{\nt{stms}}{\nt{stm}\Or{\nt{stm} \verb";" \nt{stms}}}\\
\Rule{\nt{stm}}{\nt{\dots}\Or{\nt{case}}\Or{\nt{\ldots}}} \\
\Rule{\nt{case}}{\verb"case" \nt{aexp} \verb"of" \nt{lstms} \nt{case-else}} \verb"end"\\
\Rule{\nt{case-else}}{\verb"else" \verb"(" \nt{stms} \verb")"\Or{$\epsilon$}}\\
\Rule{\nt{lstms}}{\nt{lstm}\Or{\nt{lstm} \verb"," \nt{lstms}}}\\
\Rule{\nt{lstm}}{\nt{num} \verb": (" \nt{stms} \verb")"}
\end{bnf}
where \nt{stm}, \nt{num} and \nt{aexp} have their usual connotations,
and $\epsilon$ signifies the empty production.
\begin{subquestions}
\subquestion
Sketch parsing procedures for \nt{case}, \nt{case-else},
\nt{lstms} and \nt{lstm}.
\marks{25}
\subsubquestion
Write down an abstract syntax for \nt{case} and \nt{lstm}.
\marks{20}
\subquestion
The meaning of a \nt{case} is as follows:
On entry to the \nt{case}, 
the arithmetic expression \nt{aexp} is evaluated,
delivering an integer result, and control is transferred to
the labelled \nt{lstm} which is preceded by that integer
label. If the delivered integer does not appear as a label,
and the \nt{case} contains a non-empty \nt{case-else},
control is transferred to the \nt{stms} in the \nt{case-else}.
If the delivered integer does not appear as a label,
and the \nt{case} has an empty \nt{case-else}, 
control is transferred to the statement following the \nt{case}.
\begin{subsubquestions}
\subsubquestion
	Outline and describe the structure of the
        code that might be generated in translation of
        a \nt{case}.  Assume code generation for a simple
        stack or register machine, and briefly explain the
	instructions you use in your outline.
\marks{25}
\subsubquestion
        Construct a syntax-directed translation for 
        \nt{case} which produces such code.
        Describe, in detail, any attributes you
        introduce, and make clear any assumptions you make about 
        the attributes and code generated for \nt{aexp}, 
	but do not produce translations for \nt{aexp}.
\marks{30}
\end{subsubquestions}
\end{subquestions}

\question

\begin{subquestions}
\subquestion
Draw a diagram of the usual run time memory
subdivision (memory model) for execution of
programs compiled from languages such as
Modula-2 and C.  Briefly describe the function of each
of the subdivisions (areas), and discuss
the programming language features which influence
the organisation and functioning of run time storage.
Explain why a typical C language memory model does not
have an area set aside for dynamically allocated
storage. 
\marks{30}

\subquestion
Sketch a possible structure for a stack activation
record (stack frame) and briefly explain the purpose of
each component. Comment on how local variables,
parameters and non-local variables
are addressed by the code generated for a
procedure when a run-time stack is in use.
%and explain the need for a register `spill' area
%in some records.
\marks{30}

\subquestion
Explain the purpose of the {\em heap} storage structure and
describe a simple heap allocator.  Explain what is 
meant by {\em fragmentation}, and mention some methods
for reducing fragmention in heaps.  The terms
{\em garbage\/} and {\em dangling pointer} refer to
problems which arise in programming languages which
support explicit deallocation of heap objects. Define
these terms and explain using simple example program
fragments how these problems arise.
%Sketch the simple mark-sweep garbage collection algorithm
%for automatic deallocation of inaccessible heap objects.
\marks{40}


\end{subquestions}


 
\question

\begin{subquestions}

\subquestion
Outline the Sethi-Ullman numbering algorithm and 
describe potential uses for it.  Assuming two registers
are required for loading a variable, and one for loading a literal number, 
draw the labelled
tree produced by your algorithm for the expression:
\marks{40}
\begin{verbatim}
       (a - 3) + ((c + 4) + (e * f))
\end{verbatim}

\subquestion

        Explain what is meant by the parameter passing mechanisms usually
        known as
                \begin{enumerate}
                \item call-by-value
                \item call-by-reference
                \item call-by-value-result
                \item call-by-name
                \end{enumerate}
        Deduce for each mechanism what is printed by the 
        program shown below, assuming that in each
	case all parameters are passed with the same mechanism.
        In each case briefly justify your answer.
\begin{verbatim}
       MODULE prettypass; 
       
       IMPORT InOut;

       VAR a, b: INTEGER;
 
       PROCEDURE pass (x,y,z: INTEGER);
       BEGIN
            y := y + 1;
            z := z + x
       END pass;
               
       BEGIN
            a := 2;
            b := 3;
            pass (a+b, a, a);
            InOut.WriteInt (a)
       END.
\end{verbatim}                   
Although the program is written in a Modula-2-like syntax, it
is not a valid Modula-2 program.
\marks{60}

\end{subquestions}
\end{questions}

\end{document}

