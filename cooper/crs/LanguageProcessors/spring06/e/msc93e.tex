% Herewith two questions for Prins and Tools

\question


The following is an extract from the BNF syntax for
a programming language:
\begin{bnf}
\Rule{\nt{stms}}{\nt{stm stms}\Or{\nt{stm}}} \\
\Rule{\nt{stm}}{\nt{loop}\Or{\nt{assign}}} \\
\Rule{\nt{assign}}{\nt{id} \verb":=" \nt{exp} \verb";"} \\
\Rule{\nt{loop}}{\verb"for" \nt{id} \verb"=" \nt{exp$_1$} \verb"to" \nt{exp$_2$}
 \verb"do" \nt{stms} \verb"endfor ;"} \\
\Rule{\nt{exp}}{\nt{id}\Or{\nt{integer}}\Or{\nt{exp} \verb"+" \nt{exp}}
        \Or{\nt{exp} \verb"-" \nt{exp}} \Or{\nt{exp} \verb"*" \nt{exp}}}
\end{bnf}
where \nt{id} and \nt{integer} have obvious interpretations.
\begin{subquestions}
\subsubquestion
Show that this is an ambiguous grammar by constructing different
syntax trees for the assignment:\marks{5}
\begin{verbatim}
       t := m + 60 * h
\end{verbatim}
\subquestion
Modify the grammar so that the \verb"+" and \verb"-" operators
have equal precedence lower than that of \verb"*".
\marks{4}
\subquestion
The meaning of a \nt{loop} is as follows:
\begin{itemize}
\item On entry to the loop, arithmetic expressions \nt{exp$_1$} and \nt{exp$_2$}
        are evaluated. The value of \nt{exp$_1$} is the {\em lower
        bound}, and is assigned to the identifier \nt{id}.
        The value of \nt{exp$_2$} is the {\em upper bound}.
        Neither may be changed after entry to the loop by 
        execution of \nt{stms}.
\item If the value of \nt{exp$_2$} is less than \nt{id},
        the loop body \nt{stms} is
        not executed and control transfers to the statement following
        the \verb"endfor" (leaving the loop).
\item Otherwise, the loop body \nt{stms} is
        executed,  
        then the value of \nt{id} is incremented by 1,
        and control tranfers back to the start of the loop.
\end{itemize}
\begin{subsubquestions}
\subsubquestion
	Outline and describe the structure of the
        code that might be generated in translation of
        a \nt{loop}.  Assume code generation for a simple
        stack or register machine, and briefly explain the
	instructions you use in your outline.
\marks{6}
\subsubquestion
        Construct a syntax-directed translation for 
        \nt{loop} which produces such code.
        Describe, in detail, any attributes you
        introduce, and make clear any assumptions you make about 
        the attributes and code generated for the
        expressions \nt{exp}, but do not produce translations for
	\nt{exp}.
\marks{10}
\end{subsubquestions}
\marks{???}
\end{subquestions}

\question

\begin{subquestions}
\subquestion
Draw a diagram of the usual run time memory
subdivision (memory model) for execution of
programs compiled from languages such as
Pascal and C.  Briefly describe the function of each
of the subdivisions (areas), and discuss
the programming language features which influence
the organisation and functioning of run time storage.
\marks{8}

\subquestion
Sketch a possible structure for a stack activation
record (stack frame) and briefly explain the purpose of
each component. Comment on how local variables,
parameters and non-local variables
are addressed by the code generated for a
procedure when a run-time stack is in use. 
\marks{8}

\subquestion
Outline the Sethi-Ullman numbering algorithm and 
describe potential uses for it.  Assuming two registers
are required for loading a variable, and one for loading a literal number, 
draw the labelled
tree produced by your algorithm for the expression:
\marks{9}
\begin{verbatim}
       (a - 3) + ((c + 4) + (e * f))
\end{verbatim}

\end{subquestions}
 

