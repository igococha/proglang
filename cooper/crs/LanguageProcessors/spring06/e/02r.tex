%\documentstyle[cityexam,11pt]{article}
\documentclass[11pt]{cityexam}
\usepackage{semantic}
\withmodelanswers
\begin{document}
%\examnumber{905.57}
\begin{preamble}
\degrees{B.Eng. Degree in Computing \\ B.Eng. Degree in Software Engineering}
\part{II}
\title{Language Processors}
\examdate{??? September 2002}
\examtime{???--???}
\rubric{Answer {\sl TWO} questions \\ All questions carry equal marks}
\end{preamble}
\externals{P. Hall \\ M. Moulding}
\internals{D. Bolton}

\reservestyle{\nonterm}{\textit}
\nonterm{dec,decs,vardec,fundec,id,exp,tyfields,type-id,fundecs,expseq,op}
\reservestyle{\keyword}{\textbf}
\keyword{var,function,let,in,end,if,then,else,while,do,for,break,to,switch,of,case,default}


\begin{questions}

\newpage

\question

\begin{subquestions}


\subquestion
Give a definition for regular expressions which shows how
arbitrary regular expressions can be constructed for a
given alphabet (set) of symbols. Do not include shorthand
notations you may know.
\marks{10}

\subquestion
Write down regular expressions that define a pattern to recognise
unsigned real numbers that are composed of an non-zero number of digits
followed by a decimal point (period) followed by an non-zero number of
digits. If there is more than a single digit in the string of 
digits before the point, the string must not begin with a `0'. 
If there is more than a single digit in the string of digits after the point, 
the string must not end with a `0'.
\marks{20}

\begin{modelanswer}
\begin{verbatim}
(([1-9][0-9]+)|[0-9]).([0-9]|[0-9]+[1-9])
\end{verbatim}
\end{modelanswer}

\subquestion
What is an ambiguous grammar? 
By giving suitable examples show that the
        following grammar is ambiguous. 
\begin{tabbing}
stmtxxx\=$\rightarrow$xxx\=if\kill
\it
exp \> $\rightarrow$ \> {\it exp\/} \verb!+! {\it exp} \\
\it
     \> $\mid$ \> {\it exp\/} \verb!-! {\it exp\/} \\
     \> $\mid$ \> {\it id}
\end{tabbing}
Here {\it id\/} stands for any identifier.
Explain the term {\em shift-reduce conflict\/} using as
an example the program fragment \verb!a+b-c!.
\marks{35}

\begin{modelanswer}
Two parse trees for a suitable expression (20).

Point at which a parser can either shift, or reduce by a rule. 
Here, reach the -, can shift, or reduce using first production (15).
\end{modelanswer}


\subquestion
Consider the following Tiger function:
\begin{verbatim}
1 function f(a:string, b:int, c:int)=
2       (print_int(b+c);
3        let var c := "hi"
4            var a := b
5            var b := "hello"
6        in print(b); print_int(a) 
7        end;
8        print_int(c); print_int(b);
9       )
\end{verbatim}
Given an initial environment $\sigma_0 = \{\textrm{a} \rightarrow \textrm{int}, \textrm{b} \rightarrow \textrm{string}\}$, derive the type binding environments for the function at each
use of an identifier and indicate where type lookups will occur.
\marks{35}

\begin{modelanswer}
0 $\sigma_0$ is starting environment\\
1 $\sigma_1 = \sigma_0 + \{a\rightarrow string,b\rightarrow int,c\rightarrow int\}$\\
2 lookup ids b, c  in $\sigma_1$\\
3 $\sigma_2 = \sigma_1 + \{c\rightarrow string\}$ (overrides arg c)\\
4 lookup id b, then $\sigma_3 = \sigma_2 + \{a\rightarrow int\}$ (overrides arg a)\\
5 $\sigma_4 = \sigma_3 + \{b\rightarrow string\}$ (overrides arg b\\
6 lookup id b, then a  in $\sigma_4$\\
6 discard $\sigma_4$, $\sigma_3$, $\sigma_2$ revert to $\sigma_1$\\
7 lookup c, b in $\sigma_1$\\
8 discard $\sigma_1$ revert to $\sigma_0$
\end{modelanswer}


\end{subquestions}

\newpage

\question

The reference manual for a Tiger'01-like programming language contains
the following definition for a kind of expression: 

The switch-expression
\[
\<switch> \; ( \; \<exp> \; ) \; \<of> \; ( \; \<case> \; \<exp>_{c1} \; : 
\; \<exp>_1 \; \<case> \; \<exp>_{c2} \; : \; \<exp>_2 \;  \ldots \; \<default> \; : \; \<exp>_{\it default} \; ) \;
\]
first evaluates the integer expression $\<exp>$.
This value is then compared in sequence 
with $\<exp>_{c1}, \<exp>_{c2}, \ldots $.
If there is a match for an $\<exp>_{cn}$, 
the switch-expression yields
the result of evaluating the corresponding $\<exp>_{\it n}$.
If there is no match, the switch-expression yields the
result of evaluating $\<exp>_{\it default}$.
The $\<default> \; : \; \<exp>_{\it default}$ part may be omitted, 
and then if there is no match the switch-expression produces no
value.

\begin{subquestions}
\subquestion
Write down a BNF concrete syntax for the switch-expression. 
\marks{15}

\begin{modelanswer}
Terminals in CAPS.
\begin{verbatim}
switch-exp -> SWITCH ( exp ) OF ( cases default ) 
           |  SWITCH ( exp ) OF ( cases )

default    -> DEFAULT : exp

cases      -> CASE exp : exp
           |  CASE exp : exp cases
\end{verbatim}
\end{modelanswer}

\subquestion
Sketch a possible abstract syntax for the switch-expression.
\marks{15}

\begin{modelanswer}
\begin{verbatim}
SwitchExp (Exp intexp default) (CaseList cases)
CaseList (Case head CaseList tail)
Case (Exp case resultexp) 
(be flexible about how this is expressed - it might be Java)
\end{verbatim}
\end{modelanswer}

\subquestion
Show how semantic actions in a grammar for a parser-generator such as CUP
can be used to produce abstract syntax trees for the switch-expression.
\marks{35}

\begin{modelanswer}
\begin{verbatim}
public class SwitchExp extends Exp {
   public Exp intexp, cdefault;
   public CaseList cases;
   public SwitchExp(int p, Exp e, CaseList l, Exp d) 
     {pos=p; intexp=e; cases=l; cdefault=d; }
}
public class CaseList {
   public Case head;
   public CaseList tail;
   public CaseList(Case h, CaseList t) {head=h; tail=t;}
}
public class Case extends Absyn {
   public Exp switchcase, exp;
   public Case (int p, Exp c, Exp e) 
      {pos=p; switchcase=c; exp=e;}
}
...
exp  ::= ....

     |   SWITCH:s LPAREN exp:e RPAREN OF 
           LPAREN cases:c default:d RPAREN
         {: RESULT = new Absyn.SwitchExp(sleft,e,c,d); :}

     |   SWITCH:s LPAREN exp:e RPAREN OF 
           LPAREN cases:c RPAREN
         {: RESULT = new Absyn.SwitchExp(sleft,e,c,null); :}

     ;

default ::= DEFAULT COLON exp:e {: RESULT = e ; :}
     ;

cases ::= CASE:p exp:ec COLON exp:e
          {: RESULT = new Absyn.CaseList(
                       new Absyn.Case(pleft,ec,e),null); :}
      |   CASE:p exp:ec COLON exp:e cases:m
          {: RESULT = new Absyn.CaseList(
                       new Absyn.Case(pleft,ec,e),m); :}
\end{verbatim}
\end{modelanswer}
 
\subquestion
        Outline the intermediate representation
        that might be generated from your
        asbtract syntax trees in translation of
        a switch-expression.
        You may wish to use a simple example to explain your 
        translation, eg:
\begin{verbatim}
a := switch ( x ) of 
      ( case y: x+1  case y+1: x+3  
           case y+3: x+5  default: x+10 )
\end{verbatim}
\marks{35}

\begin{modelanswer}
\begin{verbatim}
eval exp (here, x) into R1 (some TEMP loc ie a virtual reg)
eval first label exp (here, y) into R2 (some TEMP ie a reg)
CJUMP(=,R1,R2,L1,L2)
LABEL(L1)
eval first result exp (here x+1) into R3 (some TEMP)
JUMP(Lend)
LABEL(L2)
evaluate second ... (here y+1)...
...
LABEL(Ln)
eval default result exp (here x+10) into R3 (some TEMP)
JUMP(Send)
LABEL(Lend)
store result in R3 as appropriate (here into a)
LABEL(Send)
\end{verbatim}
\end{modelanswer}

\end{subquestions}

\newpage

\question


\begin{subquestions}

\subquestion
Choose a programming language you know well and describe
its run time storage requirements and functioning.
Clearly associate any storage structures you
mention with the implementation of particular 
language features.
\marks{30}

\begin{modelanswer}
Language features: procedure/method call and parameter passing,
lexical scoping, dynamic storage allocation and deallocation,
including object creation. Name stack, heap, why different.
More marks for good exposition of example language
(eg Java object creation, method call). Operation of
stack and heap (with mention of garbage collection).
\end{modelanswer}


\subquestion
What is a stack frame?
Outline a typical layout for a stack frame and describe 
each element of a frame and how it is pushed to the stack during
program execution. 
Explain the term {\em static link\/} and explain why a C or Java
implementation does not include static links in its stack frames.
%Comment on how local variables, arguments and non-local variables
%are addressed by the code generated for a procedure or method. 
\marks{40}

\begin{modelanswer}
A structure to store info local to a proc invocation.
Args, static link, locals, return address, regs/temp space.
More marks for mentioning code genned for a proc/func doing the
pushing.
%Locals, args: offsets from SP or FP; non-locals: follow access links
%(more for description of access link).
\end{modelanswer}

\subquestion
Explain why registers might be used for parameter passing and
suggest situations where passing in registers is particularly
appropriate. 
Outline situations where it is necessary for the code generated for a 
procedure or method to write registers to the stack.
\marks{30}

\begin{modelanswer}
Efficiency; particularly appropriate when leaf procs, interproc
reg alloc, dead variables, reg windows (but\ldots).
Reg saves: when address is taken,
when call-by-ref,
when accessed by inner nesting,
value too big,
an array,
convention of save for partic reg prior to call,
spilling in exp evaluation,
saving a reg window.
\end{modelanswer}

\end{subquestions}


\end{questions}

\end{document}

