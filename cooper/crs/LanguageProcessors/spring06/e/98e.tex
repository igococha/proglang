\documentstyle[cityexam,11pt]{article}
\withmodelanswers
\begin{document}
\examnumber{00.00}
\begin{preamble}
\degrees{???}
\part{II}
\title{Language Processors}
\examdate{XXX, 00 XXX, 1998}
\examtime{00 to 00}
\rubric{???}
\end{preamble}
\externals{XXX \\ XXX}
\internals{DJ Bolton}

\begin{questions}

% Herewith three questions for Language Processors

\newenvironment{bnf}{
    \begin{tabbing}
    XXXXXX \= XXXX \= XXXXXXXXXXXXXXXXXXXXXXXXXXXXXXXXXX\kill}{
    \end{tabbing}
    }

\newcommand{\nt}[1]{{\it #1\/}}
\newcommand{\Or}[1]{\ \ $|$ \ \ #1}
\newcommand{\Rule}[2]{{\it #1} \> $\rightarrow$ \> #2}

\question

The following is an extract from the BNF syntax for
an imperative programming language:
\begin{bnf}
\Rule{\nt{stms}}{\nt{stm}\Or{\nt{stm} \verb";" \nt{stms}}}\\
\Rule{\nt{stm}}{\nt{\dots}\Or{\nt{repeat}}\Or{\nt{\ldots}}} \\
\Rule{\nt{repeat}}{\verb"repeat" \nt{stms$$} \verb"until" \nt{bexp} \verb"end"}\\
\Rule{\nt{aexp}}{\nt{var}\Or{\nt{num}}\Or{\nt{aexp} \verb"+" \nt{aexp}}
        \Or{\nt{aexp} \verb"-" \nt{aexp}} \Or{\nt{aexp} \verb"*" \nt{aexp}}}\\
\Rule{\nt{bexp}}{\nt{aexp} \verb"=" \nt{aexp}\Or{\nt{aexp} \verb"<=" \nt{aexp}}}
\end{bnf}
where \nt{stm}, \nt{num} and \nt{var} have their usual connotations.
\begin{subquestions}
\subquestion
Sketch a parsing procedure for \nt{repeat}.
\marks{10}
\begin{modelanswer}
\begin{verbatim}
PROC repeat();
BEGIN
  match(Trepeat);
  stms();
  match(Tuntil);
  bexp();
  match(Tend)
END
\end{verbatim}
\end{modelanswer}

\subsubquestion
Write down an abstract syntax for \nt{repeat}, \nt{bexp} and \nt{aexp}.
\marks{10}
\begin{modelanswer}
\begin{verbatim}
repeat -> REPEAT stm bexp
bexp   -> BEQ aexp aexp | BLE aexp aexp
aexp   -> AVAR name | ANUM num | AADD aexp aexp | ASUB aexp aexp | AMUL aexp aexp
\end{verbatim}
\end{modelanswer}

\subquestion
The meaning of a \nt{repeat} is as follows:
\begin{itemize}
\item On entry to the \nt{repeat}, the \nt{stms} are executed.
\item Then the \nt{bexp} is evaluated, and if the \nt{bexp} evaluates to
	true, execution is continued at the statement
	that follows the \verb+end+ of the \nt{repeat} (leaving
	the \nt{repeat}).
\item Otherwise, when the \nt{bexp} evaluates to false,
        control transfers back to the start of the \nt{repeat}.
\end{itemize}
\begin{subsubquestions}
\subsubquestion
        Outline the structure of the
        code that might be generated in translation of
        a \nt{repeat} with an example \nt{bexp} (\verb"a+1 <= b*2", say). 
        Assume code generation for a simple stack machine, 
	and briefly explain the instructions you use in your outline.
\marks{20}
\begin{modelanswer}
\begin{verbatim}
LABEL1:
code for stms
code for RHS aexp, result on stack top
code for LHS aexp, result on stack top
LE                 top <= top - 1
JUMPFALSE LABEL1
\end{verbatim}
\end{modelanswer}

\subsubquestion
        Construct a syntax-directed translation for 
        \nt{repeat} and \nt{bexp} which produces such code.
        Describe any attributes you
        introduce, and make clear any assumptions you make about 
        the attributes and code generated for the
        expressions \nt{aexp}, but you need not produce translations for
        \nt{aexp}.
\marks{35}

\begin{modelanswer}
\begin{verbatim}
repeat -> REPEAT {
                 top := newlabel()
                 gen (LABEL top) }
            stm bexp {
                 gen (LE)
                 gen (JUMPFALSE top) }
bexp   -> BEQ aexp aexp { gen (EQ) }
       |  BLE aexp aexp { gen (LE) }
\end{verbatim}
\end{modelanswer}


\end{subsubquestions}

\subquestion

Outline the code that must be generated for procedure call and return
in a typical procedural programming language implementation.
\marks{25}
\begin{modelanswer}
\begin{verbatim}
call:
compute and pass actuals
store return address
store control link (current SP) 
increment SP
store regs if necessary
ret:
store return value
restore regs if necessary
restore control link (copy link to SP)
branch to return address
copy return value
\end{verbatim}
\end{modelanswer}


\end{subquestions}

\question

\begin{subquestions}

\subquestion
Discuss the programming language features which influence
the organisation and functioning of run time storage,
giving examples from programming languages you
have encountered. 
Comment on how local variables,
parameters and non-local variables
are addressed by the code generated for a
procedure in a typical procedural programming 
language.
\marks{35}
\begin{modelanswer}
Language features: procedure call and parameter passing,
lexical scoping, dynamic storage allocation and deallocation,
including object creation. Name stack, heap, why different.
More marks for example languages
(eg Java/Eiffel object creation, Pascal `new' etc). 
Operation of stack and heap (with mention of garbage collection).
Locals, params: offsets from SP; non-locals: follow access links
(more for description of access link).
\end{modelanswer}


\subquestion
Explain the purpose of the {\em heap} storage structure and
describe a simple heap allocator.  Explain what is 
meant by {\em fragmentation}, and mention some methods
for reducing fragmention in heaps.  
{\em Dangling pointers}
can occur in the implementation of programming languages which
support explicit deallocation of heap objects. Define
dangling pointer and explain using simple example program
fragments how they arise.
Define the term {\em garbage}, and 
sketch the simple mark-sweep garbage collection algorithm
for automatic deallocation of inaccessible heap objects.
\marks{40}
\begin{modelanswer}
Dynamic storage allocation is purpose. 
Frag: mention choose smallest, coalesce on dealloc, compact phase.
Dangling pointer: marks for prog frags.
Garbage: mark sweep.
\end{modelanswer}


\subquestion
Explain the need for a register `spill' area
in some stack activation records in a procedural
programming language implementation. 
Outline the Sethi-Ullman numbering algorithm and
describe potential uses for it.  Assuming two registers in total
are required to load a variable, and one register only
is required to load a literal number, draw the labelled
tree produced by the algorithm for the expression:
\marks{25}
\begin{verbatim}
       ((a + 4) + (b * c)) + (d - 3)
\end{verbatim}
\begin{modelanswer}
Spill area for exps which compiler detects can't be evald in
avail regs. Uses for SU: compute max regs required for proc;
decide on order of binary op;
detect when spill code needed.
\end{modelanswer}


\end{subquestions}

 
\question

\begin{subquestions}

\subquestion

What is an ambiguous grammar? By giving suitable examples show that the
        following grammar is ambiguous. 
\begin{tabbing}
stmtxxx\=$\rightarrow$xxx\=if\kill
\it
stmt \> $\rightarrow$ \> \verb"if" {\it expr\/} \verb"then" {\it stmt} \\
\it
     \> $\mid$ \> \verb"if" {\it expr\/} \verb"then" {\it stmt\/} \verb"else" {\it stmt} \\
     \> $\mid$ \> {\it other}
\end{tabbing}
Here {\it other\/} stands for any other statement.
\marks{20}
\begin{modelanswer}
Two parse trees for \verb"if b1 then if b2 then s1 else s"
\end{modelanswer}

\subquestion Rewrite the \verb"if"-\verb"then"-\verb"else" grammar in the
previous subquestion to remove the ambiguity, arranging your productions
so that the grammar matches each \verb"else" with the closest previous
unmatched \verb"then".
For example, in a parse tree for the statement:
\begin{verbatim}
if b1 then if b2 then s1 else s2
\end{verbatim}
the \verb"else" part should be associated with the 
second \verb"if"-\verb"then". 
\marks{35}
\begin{modelanswer}
\begin{verbatim}
stm         -> match-stm | unmatch-stm
match-stm   -> if exp then match-stm else match-stm | other
unmatch-stm -> if exp then stm
            |  if exp then match-stm else unmatch-stm
\end{verbatim}
\end{modelanswer}


\subquestion
Explain what is meant by the parameter passing mechanisms usually
        known as
                \begin{enumerate}
                \item call-by-value
                \item call-by-reference
                \item call-by-value-result
                \item call-by-name
                \end{enumerate}
        Deduce for each mechanism what is the effect of
        the following routine:
\begin{verbatim}
       PROCEDURE pass (p,q,r: INTEGER); 
       BEGIN
            q := q + 1;
            r := r + p
       END pass;
\end{verbatim}   
        when the sequence of assigments and a procedure call
        shown below is executed, assuming that all parameters are
        passed with the same mechanism, and that \verb"a" and
        \verb"b" are variables of type INTEGER.
        Write down what is printed in each case by the \verb"WriteInt"
        call, and clearly justify your answer by reference to your
        description of the parameter passing mechanism.
\begin{verbatim}
       a := 2;
       b := 3;
       pass (a+b,a,a);
       InOut.WriteInt (a);
\end{verbatim}
Although the program is written in Modula-2 syntax, the Modula-2
parameter-passing semantics do not apply and you must
instead consider each of the mechanisms mentioned, and assume
that it is legal in a routine to assign to a formal parameter.
\marks{45}
\begin{modelanswer}
\begin{verbatim}
cbv explanation -- answer is 2.
cbr explanation -- 
   a := a+1 = 3
   a := a+5 = 8    answer 8
cbvr explanation
   5,2,2 passed
   q := q+1 = 3
   r := r+5 = 7    answer 7 or 3 depending on copy back order
\end{verbatim}
\end{modelanswer}

\end{subquestions}

\end{questions}

\end{document}

