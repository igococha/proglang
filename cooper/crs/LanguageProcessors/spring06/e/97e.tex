\documentstyle[/homes/daveb/doc/exam_style/cityexam,11pt]{article}
\begin{document}
\examnumber{00.00}
\begin{preamble}
\degrees{???}
\part{II}
\title{Language Processors}
\examdate{XXX, 00 XXX, 1994}
\examtime{00 to 00}
\rubric{???}
\end{preamble}
\externals{XXX \\ XXX}
\internals{DJ Bolton}

\begin{questions}

% Herewith four questions for Language Processors

\newenvironment{bnf}{
    \begin{tabbing}
    XXXXXX \= XXXX \= XXXXXXXXXXXXXXXXXXXXXXXXXXXXXXXXXX\kill}{
    \end{tabbing}
    }

\newcommand{\nt}[1]{{\it #1\/}}
\newcommand{\Or}[1]{\ \ $|$ \ \ #1}
\newcommand{\Rule}[2]{{\it #1} \> $\rightarrow$ \> #2}

\question

\begin{subquestions}

\subquestion

Give a definition for regular expressions which shows how
arbitrary regular expressions can be constructed for a
given alphabet (set) of symbols. Do not include shorthand
notations you may know.
\marks{15}

\subquestion
If $A$ is a set of characters selected from an
alphabet of characters $V$, the shorthand notation {\it Not$(A)$} denotes 
$(V - A)$; that is all characters in $V$ not included in $A$.
Specify the following patterns using regular definition; 
the {\it Not\/} shorthand may be useful for some of the patterns:
\begin{subsubquestions}
\subsubquestion
A comment that begins with \verb"--" and ends with the \verb"Eol" 
(end of line) character (use \verb"Eol" to denote the end of line 
character).
\marks{10}
\subsubquestion
An identifier, composed of letters, digits, and underscores, that
begins with a letter, ends with a letter or digit, and contains
no consecutive underscores.
\marks{15}
\subsubquestion
A comment delimited by \verb"##" markers, which allows {\em single\/}
\verb"#"'s within the comment body.
\marks{20}
\end{subsubquestions}

\vspace{0.1in}

\subquestion
Study the following production from the BNF syntax for 
arithmetic expressions in an imperative programming language:
\begin{bnf}
\Rule{\nt{aexp}}{\nt{var}\Or{\nt{num}}\Or{\nt{aexp} \verb"+" \nt{aexp}}
        \Or{\nt{aexp} \verb"-" \nt{aexp}} \Or{\nt{aexp} \verb"*" \nt{aexp}}}
\end{bnf}
where \nt{var} and \nt{num} have their usual connotations.
\begin{subsubquestions}
\subsubquestion
Show that this grammar is ambiguous. Disambiguate the grammar,
arranging the productions so that 
the \verb"+" and \verb"-" operators
have equal precedence lower than that of \verb"*",
and so that expressions with the same precedence group 
left-to-right (ie \verb"2+3-4+5*6" groups \verb"(((2+3)+4)+(5*6))").
\marks{20}
\subsubquestion
        Explain why left-recursion must be eliminated from
        grammar productions which are to be used in
        construction of a recursive-descent parser.
        Write down a general rule for rewriting left-recursive
        grammar productions to be right-recursive. Use the rule
	to rewrite your answers to part i.
\marks{20}
\end{subsubquestions}

\end{subquestions}

\newpage

\question


The following is an extract from the BNF syntax for
an imperative programming language:
\begin{bnf}
\Rule{\nt{stms}}{\nt{stm}\Or{\nt{stm} \verb";" \nt{stms}}}\\
\Rule{\nt{stm}}{\nt{\dots}\Or{\nt{loop}}\Or{\nt{\ldots}}} \\
\Rule{\nt{loop}}{\verb"from" \nt{stms$_1$} \verb"until" \nt{bexp} \verb"loop" \nt{stms$_2$}} \verb"endloop"\\
\Rule{\nt{aexp}}{\nt{var}\Or{\nt{num}}\Or{\nt{aexp} \verb"+" \nt{aexp}}
        \Or{\nt{aexp} \verb"-" \nt{aexp}} \Or{\nt{aexp} \verb"*" \nt{aexp}}}\\
\Rule{\nt{bexp}}{\nt{aexp} \verb"=" \nt{aexp}\Or{\nt{aexp} \verb"<=" \nt{aexp}}}
\end{bnf}
where \nt{stm}, \nt{num} and \nt{var} have their usual connotations.
\begin{subquestions}
\subquestion
Sketch a parsing procedure for \nt{loop}.
\marks{15}
\subsubquestion
Write down an abstract syntax for \nt{loop}, \nt{bexp} and \nt{aexp}.
\marks{15}
\subquestion
The meaning of a \nt{loop} is as follows:
\begin{itemize}
\item The effect of a \nt{Loop} is the effect of executing 
        \nt{stms$_1$} followed by the rest of the loop.
\item The effect of executing the rest of the loop
        \verb|until| \ldots\ \verb|end|
        is to leave the
        state of the computation unchanged if the \nt{bexp}
        evaluates to true.
\item Otherwise, when the \nt{bexp} evaluates to false,
        it is the effect of executing \nt{stms$_2$},
        followed by the effect of executing the rest of the loop
        \verb|until| \ldots\ \verb|end| again in the resulting state.
\end{itemize}
\begin{subsubquestions}
\subsubquestion
        Outline the structure of the
        code that might be generated in translation of
        a \nt{loop} with an example \nt{bexp} (\verb"a+1 <= b*2", say). 
        Assume code generation for a simple
        stack machine, and briefly explain the
        instructions you use in your outline.
\marks{25}
\subsubquestion
        Construct a syntax-directed translation for 
        \nt{loop} and \nt{bexp} which produces such code.
        Describe any attributes you
        introduce, and make clear any assumptions you make about 
        the attributes and code generated for the
        expressions \nt{aexp}, but you need not produce translations for
        \nt{aexp}.
\marks{45}
\end{subsubquestions}
\end{subquestions}

\question

\begin{subquestions}

\subquestion
Sketch a possible structure for a stack activation
record (stack frame) and briefly explain the purpose of
each component. Comment on how local variables,
parameters and non-local variables
are addressed by the code generated for a
procedure when a run-time stack is in use,
and explain the need for a register `spill' area
in some records.
\marks{30}

\subquestion
Explain the purpose of the {\em heap} storage structure and
describe a simple heap allocator.  Explain what is 
meant by {\em fragmentation}, and mention some methods
for reducing fragmention in heaps.  
{\em Garbage\/} and {\em dangling pointers}
can occur in the implementation of programming languages which
support explicit deallocation of heap objects. Define
these terms and explain using simple example program
fragments how they arise.
%Sketch the simple mark-sweep garbage collection algorithm
%for automatic deallocation of inaccessible heap objects.
\marks{40}

\subquestion
Outline the Sethi-Ullman numbering algorithm and
describe potential uses for it.  Assuming two registers
are required for loading a variable, and one for loading a literal number,
draw the labelled
tree produced by your algorithm for the expression:
\marks{30}
\begin{verbatim}
       ((a + 4) + (b * c)) + (d - 3)
\end{verbatim}

\end{subquestions}

 
\question

\begin{subquestions}

\subquestion
Draw a diagram of the usual run time memory
subdivision (memory model) for execution of
programs compiled from languages such as
Modula-2 and C.  
%Briefly describe the function of each
%of the subdivisions (areas), and 
Discuss the programming language features which influence
the organisation and functioning of run time storage,
clearly associating the memory subdivisions you 
mention with the implementation of particular language features.
Choose a language you know well and briefly describe
its run time storage requirements and functioning.
%Explain why a typical C language memory model does not
%have an area set aside for dynamically allocated
%storage.
\marks{40}

\subquestion

        Explain what is meant by the parameter passing mechanisms usually
        known as
                \begin{enumerate}
                \item call-by-value
                \item call-by-reference
                \item call-by-value-result
%                \item call-by-name
                \end{enumerate}
        Deduce for each mechanism what is printed by the 
        program shown below, assuming that in each
	case both parameters are passed with the same mechanism.
        In each case you must clearly justify your answer.
\begin{verbatim}
       MODULE prettypass;
       
       IMPORT InOut;

       VAR i, j: INTEGER;

       PROCEDURE mult (m,n: INTEGER);
       BEGIN
            m := m * n;
            InOut.WriteInt (m);
            InOut.WriteInt (n)
       END mult;
               
       BEGIN
            i := 2;
            j := 3;
            mult (i, j)
            mult (i, i)
       END.
\end{verbatim}                   
Although the program is written in a Modula-2-like syntax, 
the Modula-2 parameter-passing semantics do not apply and you must
instead consider each of the mechanisms mentioned.
\marks{60}

\end{subquestions}
\end{questions}

\end{document}

