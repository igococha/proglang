\documentstyle[/homes/daveb/doc/exam_style/cityexam,11pt]{article}
\begin{document}
\examnumber{00.00}
\begin{preamble}
\degrees{EC2 retread}
\part{II}
\title{Programming Systems and Principles of Programming Languages}
\examdate{XXX, 00 XXX, 1994}
\examtime{00 to 00}
\rubric{Two from section A, two from B}
\end{preamble}
\externals{XXX \\ XXX}
\internals{S Hunt \\ DJ Bolton}

\begin{questions}

% Herewith two questions for Resit Lang Processors

\question

\newenvironment{bnf}{
    \begin{tabbing}
    XXXXXXX \= XXXX \= XXXXXXXXXXXXXXXXXXXXXXXXXXXXXXXXXX\kill}{
    \end{tabbing}
    }

\newcommand{\nt}[1]{{\it #1\/}}
\newcommand{\Or}[1]{\ \ $|$ \ \ #1}
\newcommand{\Rule}[2]{{\it #1} \> $\rightarrow$ \> #2}

The following is an extract from the BNF syntax for
a programming language:
\begin{bnf}
\Rule{\nt{stms}}{\nt{stm stms}\Or{\nt{stm}}} \\
\Rule{\nt{stm}}{\nt{loop}\Or{\nt{assign}}} \\
\Rule{\nt{assign}}{\nt{id} \verb":=" \nt{exp} \verb";"} \\
\Rule{\nt{loop}}{\verb"for" \nt{id} \verb"=" \nt{exp$_1$} \verb"to" \nt{exp$_2$}
 \verb"do" \nt{stms} \verb"endfor ;"} \\
\Rule{\nt{exp}}{\nt{id}\Or{\nt{integer}}\Or{\nt{exp} \verb"+" \nt{exp}}
        \Or{\nt{exp} \verb"-" \nt{exp}} \Or{\nt{exp} \verb"*" \nt{exp}}}
\end{bnf}
where \nt{id} and \nt{integer} have obvious interpretations.
\begin{subquestions}
\subsubquestion
Show that this is an ambiguous grammar by constructing different
syntax trees for the assignment:\marks{20}
\begin{verbatim}
       t := m + 60 * h
\end{verbatim}
%\subquestion
%Modify the grammar so that the \verb"+" and \verb"-" operators
%have equal precedence lower than that of \verb"*".
%\marks{4}
\subquestion
The meaning of a \nt{loop} is as follows:
\begin{itemize}
\item On entry to the loop, arithmetic expressions \nt{exp$_1$} and \nt{exp$_2$}
        are evaluated. The value of \nt{exp$_1$} is the {\em lower
        bound}, and is assigned to the identifier \nt{id}.
        The value of \nt{exp$_2$} is the {\em upper bound}.
        Neither may be changed after entry to the loop by 
        execution of \nt{stms}.
\item If the value of \nt{exp$_2$} is less than \nt{id},
        the loop body \nt{stms} is
        not executed and control transfers to the statement following
        the \verb"endfor" (leaving the loop).
\item Otherwise, the loop body \nt{stms} is
        executed,  
        then the value of \nt{id} is incremented by 1,
        and control tranfers back to the start of the loop.
\end{itemize}

\begin{subsubquestions}
\subsubquestion
        Outline and describe the structure of the
        code that might be generated in translation of
        a \nt{loop}.  Assume code generation for a simple
        stack or register machine, and briefly explain the
        instructions you use in your outline.
\marks{30}
\subsubquestion
        Construct a syntax-directed translation for 
        \nt{loop} which produces such code.
        Describe, in detail, any attributes you
        introduce, and make clear any assumptions you make about 
        the attributes and code generated for the
        expressions \nt{exp}, but do not produce translations for
        \nt{exp}.
\marks{50}
\end{subsubquestions}
\end{subquestions}

\question

\begin{subquestions}
\subquestion
Draw a diagram of the usual run time memory
subdivision (memory model) for execution of
programs compiled from languages such as
Pascal and C.  Briefly describe the function of each
of the subdivisions (areas), and discuss
the programming language features which influence
the organisation and functioning of run time storage.
Comment on how local variables,
parameters and non-local variables
are addressed by the code generated for a
procedure when a run-time stack is in use.
\marks{30}


\subquestion

        Explain what is meant by the parameter passing mechanisms usually
        known as
                \begin{enumerate}
                \item call-by-value
                \item call-by-reference
                \item call-by-value-result
%                \item call-by-name
                \end{enumerate}
        Deduce for each mechanism what is printed by the 
	following routine:
\begin{verbatim}
       PROCEDURE mult (m, n: INTEGER); 
       BEGIN
          m := m * n;
          InOut.WriteInt (m);
          InOut.WriteInt (n)
       END; 
\end{verbatim}   
	for the sequence of assigments and two procedure calls
        shown below, assuming that both parameters are
	passed with the same mechanism, and that \verb"i" and
	\verb"j" are variables attributes of type \verb"INTEGER".
        In each case you must clearly justify your answer.
\begin{verbatim}
       i := 2;
       j := 3;
       mult (i, j);
       mult (i, i)
\end{verbatim}
Although the program is written in Modula-2-like syntax, the 
Modula-2 parameter-passing semantics do not apply and you must
instead consider each of the mechanisms mentioned, and assume
that it is legal in a routine to assign to a formal parameter.
\marks{70}

\end{subquestions}

\question

\begin{subquestions}

\subquestion
Explain the purpose of the {\em heap} storage structure and
describe a simple heap allocator.  Explain what is 
meant by {\em fragmentation}, and mention some methods
for reducing fragmention in heaps.  
%The terms
%{\em garbage\/} and {\em dangling pointer} refer to
%problems which arise in programming languages which
%support explicit deallocation of heap objects. Define
%these terms and explain using simple example program
%fragments how these problems arise.
Sketch the simple mark-sweep garbage collection algorithm
for automatic deallocation of inaccessible heap objects.
\marks{50}

\subquestion
Assume we have the following import and pointer variable
declarations, where \verb"ptrTYPE" is a POINTER type for pointers
to objects of type \verb"TYPE":
\begin{verbatim}
IMPORT Storage
...
VAR ptr1, ptr2: ptrTYPE;
\end{verbatim}
Identify the particular problem, 
associated 
with programming languages
which support explicit deallocation of heap objects,
which is caused by the following code examples:
\begin{subsubquestions}
\subsubquestion
%Dangling pointer:
\begin{verbatim}
   Storage.Allocate (ptr1,SIZE(TYPE));
   ...some code involving ptr1...
   ptr2 := ptr1;
   Storage.DeAllocate (ptr2,SIZE(TYPE));
   ...further code...
\end{verbatim}
Hint: consider uses of \verb"ptr1" within \verb"...further code...".
\marks{25}
\subsubquestion
%Garbage:
\begin{verbatim}
   Storage.Allocate (ptr1,SIZE(TYPE));
   Storage.Allocate (ptr2,SIZE(TYPE));
   ...code involving ptr1 and ptr2...
   ptr2 := ptr1;
   ...further code...
\end{verbatim}
Hint: consider what happens to the space allocated for 
\verb"ptr2" during execution of \verb"...further code...".
\marks{25}
\end{subsubquestions}
In each case give the usual term for the problem and 
discuss the effects of the problem.

\end{subquestions}


\end{questions}

\end{document}

