%\documentstyle[cityexam,11pt]{article}
\documentclass[11pt]{cityexam}
\usepackage{semantic}
%\withmodelanswers
\begin{document}
\examnumber{510.00}
\begin{preamble}
\degrees{B.Eng. Degree in Computing \\ B.Eng. Degree in Software Engineering}
\part{II}
\title{Language Processors}
\examdate{10th May 2002}
\examtime{0900--1030}
\rubric{Answer {\sl TWO} questions \\ All questions carry equal marks}
\end{preamble}
\externals{Professor M.E.C\@. Hull\\ Professor M\@. Moulding}
\internals{D. Bolton}

\reservestyle{\nonterm}{\textit}
\nonterm{dec,decs,vardec,fundec,id,exp,tyfields,type-id,fundecs,expseq,op}
\reservestyle{\keyword}{\textbf}
\keyword{var,function,let,in,end,if,then,else,while,do,for,break,to,switch,of,case,default}


\begin{questions}


\question

\begin{subquestions}


\subquestion
Give a definition for regular expressions which shows how
arbitrary regular expressions can be constructed for a
given alphabet (set) of symbols. Do not include shorthand
notations you may know.
\marks{10}

\subquestion
Write down regular expressions that define a pattern to recognise 
a decimal literal. Decimal literals start with a numeral consisting of
one or more digits. This may be followed by an optional decimal part consisting
of a decimal point `.' followed by another numeral consisting of
one or more digits. Finally this may be followed by an optional exponent
consisting of the letter `E' followed by an optional sign followed 
by a numeral consisting of one or more digits.
\marks{25}

\begin{modelanswer}
\begin{verbatim}
[0-9]+(.[0-9]+)?(E[-+]?[0-9]+)?
\end{verbatim}
More marks for breaking it up and making readable(!).
\end{modelanswer}


\subquestion
Write down regular expressions that define a pattern to recognise 
identifiers that begin with a letter, 
and may be followed by any number of
letters, digits, and underscores. 
Sequences of underscores are not permitted,
and the last character cannot be an underscore.
\marks{30}

\begin{modelanswer}
\begin{verbatim}
[A-Za-z](_?[A-Za-z0-9])*
\end{verbatim}
\end{modelanswer}



\subquestion
Consider the following Tiger function:
\begin{verbatim}
1 function f(a:string, b:int, c:int)=
2       (print_int(b+c);
3        let var a := b
4            var b := "hello"
5        in print(b); print_int(a) 
6        end;
7        print_int(b);
8       )
\end{verbatim}
Given an initial environment $\sigma_0 = \{\textrm{a} \rightarrow \textrm{int}, \textrm{b} \rightarrow \textrm{string}\}$, 
derive the type binding environments for the function at each
use of an identifier and indicate in which type environment
type lookups will occur.
\marks{35}

\begin{modelanswer}
0 $\sigma_0$ is starting environment\\
1 $\sigma_1 = \sigma_0 + \{a\rightarrow string,b\rightarrow int,c\rightarrow int\}$\\
2 lookup ids b, c  in $\sigma_1$\\
3 lookup id b, then $\sigma_2 = \sigma_1 + \{a\rightarrow int\}$ (overrides arg a)\\
4 $\sigma_3 = \sigma_2 + \{b\rightarrow string\}$ (overrides arg b\\
5 lookup id b, then a in $\sigma_3$\\
6 discard $\sigma_3$, $\sigma_2$, revert to $\sigma_1$\\
7 lookup b in $\sigma_1$\\
8 discard $\sigma_1$ revert to $\sigma_0$
\end{modelanswer}


\end{subquestions}

\newpage

\question

The reference manual for a Tiger'01-like programming language contains
the following definition for a kind of expression: 

The for-expression
\[
\<for> \; \<id> \; = \; \<exp>_1 \; \<to> \; \<exp>_2 \; \<do> \; \<exp>_3
\] 
iterates $\<exp>_3$ over each integer value of \<id> from the
value of $\<exp>_1$ to that of $\<exp>_2$.
The variable \<id> is a new variable implicitly declared by the
\<for> statement, whose
scope covers only $\<exp>_3$,
and may not be assigned to. 
The lower and upper bounds $\<exp>_1$ and $\<exp>_2$
are evaluated only once,
prior to entering the body of the loop $\<exp>_3$.
If the upper bound is less than the lower, the
body is not executed.


\begin{subquestions}
\subquestion
Write down a BNF concrete syntax for the for-expression. 
\marks{10}

\begin{modelanswer}
Terminals in CAPS (trivial).
\begin{verbatim}
for-exp -> FOR id = exp TO exp DO exp
\end{verbatim}
\end{modelanswer}

\subquestion
Sketch a possible abstract syntax for the for-expression.
\marks{10}

\begin{modelanswer}
\begin{verbatim}
ForExp id (Exp lowerbound upperbound body) 
(be flexible about how this is expressed - it might be Java)
\end{verbatim}
\end{modelanswer}

\subquestion
Show how semantic actions in a grammar for a parser-generator such as CUP
can be used to produce abstract syntax trees for the for-expression.
\marks{30}

\begin{modelanswer}
\begin{verbatim}
        |   FOR:f ID:i ASSIGN exp:e1 TO exp:e2 DO exp:e3
            {: RESULT = new Absyn.ForExp(fleft, 
                  new Absyn.VarDec(ileft,sym(i),
                          inttype,e1),e2,e3); :}
\end{verbatim}
\end{modelanswer}

\subquestion
Explain how the scope rules for the implicitly declared variable might
be implemented.
\marks{15}

\begin{modelanswer}
By pushing a new scope for the for-loop itself. More marks if consideration
given to not assigning to it.
\end{modelanswer}

\subquestion
        Outline the intermediate representation
        that might be generated from your
	abstract syntax trees in translation of
        a for-expression.
	You may wish to use a simple example to explain your 
	translation, eg:
\begin{verbatim}
for i = j to j+10 do ( x := i*i; print(x) )
\end{verbatim}
\marks{35}

\begin{modelanswer}
\begin{verbatim}
eval exp1 (here, j) into R1 (some TEMP loc ie a virtual reg)
eval exp2 (here, j+10) into R2 (some TEMP ie a reg)
MOVE(MEM(i),R1)
MOVE(MEM(i+16),R2)
LABEL(Lstart)
MOVE(R1,MEM(i))
MOVE(R2,MEM(i+16))
CJUMP(<,R2,R1,Lend,Lbody)
LABEL(Lbody)
code for body (here square and print)
MOVE(MEM(i),BINOP(+,MEM(i),1))
JUMP(Lstart)
LABEL(Lend)
\end{verbatim}
\end{modelanswer}

\end{subquestions}

\question


\begin{subquestions}

\subquestion
Choose a programming language you know well and describe
how run-time storage is organised and managed during program 
execution.
Clearly associate any storage structures you
mention with the implementation of particular 
language features.
\marks{35}

\begin{modelanswer}
Language features: procedure/method call and parameter passing,
lexical scoping, dynamic storage allocation and deallocation,
including object creation. Name stack, heap, why different.
More marks for good exposition of example language
(eg Java object creation, method call). Operation of
stack and heap (with mention of garbage collection).
\end{modelanswer}


\subquestion
What is a stack frame?
Outline a typical layout for a stack frame and describe 
each element of a frame and how it is pushed to the stack during
program execution. 
%Explain the term {\em static link\/} and explain why a C or Java
%implementation does not include static links in its stack frames.
Comment on how local variables, arguments and non-local variables
are addressed by the code generated for a procedure or method. 
\marks{35}

\begin{modelanswer}
A structure to store info local to a proc invocation.
Args, static link, locals, return address, regs/temp space.
More marks for mentioning code genned for a proc/func doing the
pushing.
Locals, args: offsets from SP or FP; non-locals: follow static links
(more for description of static link).
\end{modelanswer}

\subquestion
Explain why registers might be used for parameter passing and
suggest situations where passing in registers is particularly
appropriate. 
Outline situations where it is necessary for the code generated for a 
procedure or method to write registers to the stack.
\marks{30}

\begin{modelanswer}
Efficiency; particularly appropriate when leaf procs, interproc
reg alloc, dead variables, reg windows (but\ldots).
Reg saves: when address is taken,
when call-by-ref,
when accessed by inner nesting,
value too big,
an array,
convention of save for partic reg prior to call,
spilling in exp evaluation,
saving a reg window.
\end{modelanswer}

\end{subquestions}


\end{questions}

\end{document}

