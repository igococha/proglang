%\documentstyle[cityexam,11pt]{article}
\documentclass[11pt]{cityexam}
\usepackage{semantic}
%\withmodelanswers
\begin{document}
%\examnumber{xxx}
\begin{preamble}
\degrees{B.Eng. Degree in Computing \\ B.Eng. Degree in Software Engineering}
\part{II}
\title{Language Processors}
\examdate{18th May 2001}
\examtime{1.00--2.30pm}
\rubric{Answer {\sl TWO} questions \\ All questions carry equal marks}
\end{preamble}
\externals{P. Hall \\ M. Moulding}
\internals{D. Bolton}

\reservestyle{\nonterm}{\textit}
\nonterm{dec,decs,vardec,fundec,id,exp,tyfields,type-id,fundecs,expseq,op}
\reservestyle{\keyword}{\textbf}
\keyword{var,function,let,in,end,if,then,else,while,do,for,break,to,switch,of,case,default}


\begin{questions}

% Herewith three questions for Language Processors
% Modified after the following comments from Pat Hall:
%P211: Language Processors I
%1.  The questions seem very mechanical without graded parts for weaker and
%stronger students.
%% IS THIS REALLY TRUE?
%2.  Question 1 (a) seems very odd and unclear, and really should be
%replaced.  The claimed difference between the editor pattern and a regular
%expression seems to hinge on what a "character" is.  This really is not the
%basis for a question.  Why not just ask about regular expressions for an
%easy first part for all students to score on.
%% DONE.
%3. Q1  (c) - I assume that "Tiger" is a programming language with which the
%students are familiar.
%% IT IS.
%4.  Question 2 (a) answer does not adequately distinguish between terminals
%and non-terminals - would you accept this sort of confusion from a student?
%% DONE.
%5.  Some of the sample answers are just too vague for me to make any
%reasonable assessment - for example Question 3.
%% THEY'RE MOSTLY BOOKWORK - SHOULD I BE DOING MORE THAN THIS UNDER OUR 
%% GUIDELINES?

\question

\begin{subquestions}

%\subquestion
%        In some editors and shells, pattern matching expression
%	notation is provided as detailed in the table below.
%        For example, the expression \verb"{.*}" 
%        matches any sequence of characters with enclosing
%        curly brackets.
%        Show how the patterns in the table can be represented by regular
%        expressions. Show also that regular expression notation is
%        more powerful than the editor pattern notation by
%        finding examples of patterns that can be expressed in the
%        regular expression notation but not in the editor pattern
%        notation.
%
%%\vspace{0.2in}
%
%\begin{center}
%\begin{tabular}{|c|c|c|}        \hline
%Expression & Matches & Examples \\ \hline
%\verb"c"                & single character \verb"c"      & \verb"c", \verb"x", \verb"!"\\                     
%\verb"*"                & zero or more characters        & \verb"lo*p" matches \verb"lp", \verb"lop", \verb"loop", \verb"looop"\ldots\\
%\verb"."                & any character                  & \verb"c.t" matches \verb"cat", \verb"cct", \verb"c%t"\ldots\\
%\verb"["{\it s\/}\verb"]" & any character in {\it s}     & \verb"c[aou]t"  matches \verb"cat", \verb"cot", \verb"cut"\\
% \hline
%\end{tabular}
%\end{center}
%\marks{30}
%
%\begin{modelanswer}
%\begin{verbatim}
%c -- c   3\\
%* -- a*, a in alphabet    5\\
%. -- a1 or a2 \ldots for all a in alphabet    5 \\
%\[s\] -- a1 or a2 \ldots for all a in s    7
%
%* in editor constrained to work on characters; alternates in
%\[..\] may only be characters (10).
%%\end{verbatim}
%\end{modelanswer}  

\subquestion
Give a definition for regular expressions which shows how
arbitrary regular expressions can be constructed for a
given alphabet (set) of symbols. Do not include shorthand
notations you may know.
\marks{10}

\subquestion
Write down regular expressions that define a pattern to recognise
identifiers that are composed of letters, digits, and underscores, that begin
begins with a letter, end with a letter or digit, and contain
no consecutive underscores.
\marks{20}
\begin{modelanswer}
\begin{verbatim}
L((L|D)*(_(L|D)+)*      L=[a-zA-Z] D=[0-9]
\end{verbatim}
\end{modelanswer}

\subquestion
What is an ambiguous grammar? By giving suitable examples show that the
        following grammar is ambiguous. 
\begin{tabbing}
stmtxxx\=$\rightarrow$xxx\=if\kill
\it
stmt \> $\rightarrow$ \> \textbf{if} {\it expr\/} \textbf{then} {\it stmt} \\
\it
     \> $\mid$ \> \textbf{if} {\it expr\/} \textbf{then} {\it stmt\/} \textbf{else} {\it stmt} \\
     \> $\mid$ \> {\it other}
\end{tabbing}
Here {\it other\/} stands for any other statement.
Explain the term {\em shift-reduce conflict\/} using as
an example the program fragment \verb+if a then if b then s1 else s2+.
\marks{35}

\begin{modelanswer}
Two parse trees for \verb"if a then if b then s1 else s2" (15).

Point at which a parser can either shift, or reduce by a rule. 
Here, reach the else, can shift, or reduce using first production (15).
\end{modelanswer}


\subquestion
Consider the following Tiger function:
\begin{verbatim}
1 function f(a:string, b:int, c:int)=
2       (print_int(b+c); print(a);
3        let var j := c
4            var b := "hello"
5        in print(b); print_int(j);
6        end;
7        print_int(b);
8       )
\end{verbatim}
Given an initial environment $\sigma_0 = \{\textrm{a} \rightarrow \textrm{int}, \textrm{b} \rightarrow \textrm{string}\}$, derive the type binding environments for the function at each
use of an identifier and indicate where type lookups will occur.
\marks{35}

\begin{modelanswer}
0 $\sigma_0$ is starting environment\\
1 $\sigma_1 = \sigma_0 + \{a\rightarrow string,b\rightarrow int,c\rightarrow int\}$\\
2 lookup ids b, c  and then a, in $\sigma_1$\\
3 lookup id c, then $\sigma_2 = \sigma_1 + \{j\rightarrow int\}$\\
4 $\sigma_3 = \sigma_2 + \{b\rightarrow string\}$ (overrides arg b)\\
5 lookup id b, then j  in $\sigma_3$\\
6 discard $\sigma_3$, $\sigma_2$ revert to $\sigma_1$\\
7 lookup b in $\sigma_1$\\
8 discard $\sigma_1$ revert to $\sigma_0$
\end{modelanswer}


\end{subquestions}

\newpage

\question

The reference manual for a Tiger'01-like programming language contains
the following definition for a kind of expression: 

The switch-expression
\[
\<switch> \; ( \; \<exp> \; ) \; \<of> \; ( \; \<case> \; \<exp>_{c1} \; : \; \<exp>_1 \; \<case> \; \<exp>_{c2} \; : \; \<exp>_2 \;  \ldots \; \<default> \; : \; \<exp>_{\it default} \; ) \;
\]
first evaluates the integer expression $\<exp>$.
This value is then compared in sequence 
with $\<exp>_{c1}, \<exp>_{c2}, \ldots $.
If there is a match for an $\<exp>_{cn}$, 
the switch-expression yields
the result of evaluating the corresponding $\<exp>_{\it n}$.
If there is no match, the switch-expression yields the
result of evaluating $\<exp>_{\it default}$.
The $\<default> \; : \; \<exp>_{\it default}$ part may be omitted, 
and then if there is no match the switch-expression produces no
value.

\begin{subquestions}
\subquestion
Write down a BNF concrete syntax for the switch-expression. 
\marks{15}

\begin{modelanswer}
Terminals in CAPS.
\begin{verbatim}
switch-exp -> SWITCH ( exp ) OF ( cases default ) 
           |  SWITCH ( exp ) OF ( cases )

default    -> DEFAULT : exp

cases      -> CASE exp : exp
           |  CASE exp : exp cases
\end{verbatim}
\end{modelanswer}

\subquestion
Sketch a possible abstract syntax for the switch-expression.
\marks{15}

\begin{modelanswer}
\begin{verbatim}
SwitchExp (Exp intexp default) (CaseList cases)
CaseList (Case head CaseList tail)
Case (Exp case resultexp) 
(be flexible about how this is expressed - it might be Java)
\end{verbatim}
\end{modelanswer}

\subquestion
Show how semantic actions in a grammar for a parser-generator such as CUP
can be used to produce abstract syntax trees for the switch-expression.
\marks{35}

\begin{modelanswer}
\begin{verbatim}
public class SwitchExp extends Exp {
   public Exp intexp, cdefault;
   public CaseList cases;
   public SwitchExp(int p, Exp e, CaseList l, Exp d) 
     {pos=p; intexp=e; cases=l; cdefault=d; }
}
public class CaseList {
   public Case head;
   public CaseList tail;
   public CaseList(Case h, CaseList t) {head=h; tail=t;}
}
public class Case extends Absyn {
   public Exp switchcase, exp;
   public Case (int p, Exp c, Exp e) 
      {pos=p; switchcase=c; exp=e;}
}
...
exp  ::= ....

     |   SWITCH:s LPAREN exp:e RPAREN OF 
           LPAREN cases:c default:d RPAREN
         {: RESULT = new Absyn.SwitchExp(sleft,e,c,d); :}

     |   SWITCH:s LPAREN exp:e RPAREN OF 
           LPAREN cases:c RPAREN
         {: RESULT = new Absyn.SwitchExp(sleft,e,c,null); :}

     ;

default ::= DEFAULT COLON exp:e {: RESULT = e ; :}
     ;

cases ::= CASE:p exp:ec COLON exp:e
          {: RESULT = new Absyn.CaseList(
                       new Absyn.Case(pleft,ec,e),null); :}
      |   CASE:p exp:ec COLON exp:e cases:m
          {: RESULT = new Absyn.CaseList(
                       new Absyn.Case(pleft,ec,e),m); :}
\end{verbatim}
\end{modelanswer}
 
\subquestion
        Outline the intermediate representation
        that might be generated from your
	asbtract syntax trees in translation of
        a switch-expression.
	You may wish to use a simple example to explain your 
	translation, eg:
\begin{verbatim}
a := switch ( x ) of 
      ( case y: x+1  case y+1: x+3  
           case y+3: x+5  default: x+10 )
\end{verbatim}
\marks{35}

\begin{modelanswer}
\begin{verbatim}
eval exp (here, x) into R1 (some TEMP loc ie a virtual reg)
eval first label exp (here, y) into R2 (some TEMP ie a reg)
CJUMP(=,R1,R2,L1,L2)
LABEL(L1)
eval first result exp (here x+1) into R3 (some TEMP)
JUMP(Lend)
LABEL(L2)
evaluate second ... (here y+1)...
...
LABEL(Ln)
eval default result exp (here x+10) into R3 (some TEMP)
JUMP(Send)
LABEL(Lend)
store result in R3 as appropriate (here into a)
LABEL(Send)
\end{verbatim}
\end{modelanswer}

\end{subquestions}

\question


\begin{subquestions}

\subquestion
%Briefly describe the function of each
%of the subdivisions (areas), and
Discuss the programming language features which influence
the organisation and functioning of run time storage,
clearly associating any storage structures you
mention with the implementation of particular language features.
Choose a language you know well and briefly describe
its run time storage requirements and functioning.
\marks{35}

\begin{modelanswer}
Language features: procedure/method call and parameter passing,
lexical scoping, dynamic storage allocation and deallocation,
including object creation. Name stack, heap, why different.
More marks for example languages
(eg Java object creation, method call). Operation of
stack and heap (with mention of garbage collection).
\end{modelanswer}


\subquestion
Outline a typical layout for a stack frame and describe 
each element of a frame and how it is pushed to the stack during
program execution. 
Comment on how local variables, arguments and non-local variables
are addressed by the code generated for a procedure or method. 
\marks{35}

\begin{modelanswer}
Args, static link, locals, return address, regs/temp space.
Locals, args: offsets from SP or FP; non-locals: follow access links
(more for description of access link).
\end{modelanswer}

\subquestion
Explain why registers might be used for parameter passing and
suggest situations where passing in registers is particularly
appropriate. 
Describe cases where it is necessary for a code generator
for a procedure or method to produce code that writes 
register values to the stack. 
\marks{30}

\begin{modelanswer}
Efficiency; particularly appropriate when leaf procs, interproc
reg alloc, dead variables, reg windows (but\ldots).
Reg saves: when address is taken,
when call-by-ref,
when accessed by inner nesting,
value too big,
an array,
convention of save for partic reg prior to call,
spilling in exp evaluation,
saving a reg window.
\end{modelanswer}

\end{subquestions}


\end{questions}

\end{document}

