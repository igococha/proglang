%\documentstyle[cityexam,11pt]{article}
\documentclass[11pt]{bareexam}
\usepackage{semantic}
%\withmodelanswers
\begin{document}
%\examnumber{???}
%\begin{preamble}
%\degrees{???}

%\part{II}
%\title{Language Processors}
%\examdate{???}
%\examtime{???}
%\rubric{Answer {\sl TWO} questions \\ All questions carry equal marks}
%\end{preamble}
%\externals{???}
\internals{\ }

\reservestyle{\nonterm}{\textit}
\nonterm{Statement,Exp,dec,decs,vardec,fundec,id,exp,tyfields,type-id,fundecs,expseq,op}
\reservestyle{\keyword}{\textbf}
\keyword{var,function,let,in,end,if,then,else,while,do,for,break,to,switch,of,case,default}


\begin{questions}

\question

The reference manual for a MiniJava-like programming language contains
the following grammar for a for-statement: 

\[
\<Statement> \; -> \; \<for> \; ( \<id> = \<Exp> \; ; \; \<Exp> \; ; \; \<id> = 
\<Exp> ) \; \<Statement>
\]


\begin{subquestions}

\subquestion
Sketch a possible abstract syntax for the for-statement.
\marks{15}

\begin{modelanswer}
\begin{verbatim}
For Id Exp Exp Id Exp Statement
(be flexible about how this is expressed - it might be Java and have types)
\end{verbatim}
\end{modelanswer}

\subquestion
Show how semantic actions in a grammar for a parser-generator such as JavaCC
can be used to produce abstract syntax trees for the for-statement. 
\marks{25}

\begin{modelanswer}
\begin{verbatim}
Statement ForStatement() : 
{ Identifier i1, i2;
  Exp e1, e2, e3;
  Statement s;
}
{
  "for" "(" i1=Identifier() "=" e1=Expression() ";"
            e2=Expression() ";"
            i2=Identifier() "=" e3=Expression() 
         ")" s=Statement() 
  { return new For(i1,e1,e2,i2,e3,s); }
}
\end{verbatim}
\end{modelanswer}

\subquestion
Informally describe an appropriate typecheck for the for-statement.
\marks{10}

\begin{modelanswer}
i1 should be same type as e1, i2 same type as e3;
e2 must be a boolean expression. 
\end{modelanswer}

\subquestion


Suppose a compiler for a MiniJava-like language that includes
a for-statement translates all statements and expressions
into intermediate code (eg intermediate representation (IR) trees).

\begin{subsubquestions}

\subsubquestion
Draw or write down
the intermediate representation
required to access a local variable declared in a method.
Explain your answer.
\marks{15}

\begin{modelanswer}
MEM(BINOP(PLUS,TEMP fp, CONST k)) where k is offset of var in frame, fp the
register holding the framepointer. Has to compute place in frame.
\end{modelanswer}

\subsubquestion
Outline the intermediate code that might be generated
in translation of the for-statement. You may wish to use a simple
example to explain your translation, eg:
\begin{verbatim}
for (i = j; i < k; i=i+1) 
   { x = i*i; System.out.println (x); }
\end{verbatim}
You can assume that the expression tree for any variable \verb"v" is
simply \verb"TEMP v". Do not show translations for the body of the
example for-statement (in braces in this example \verb+{...}+).
\marks{35}

\begin{modelanswer}
\begin{verbatim}
MOVE(TEMP i,TEMP j)
LABEL(Lstart)
CJUMP(<,TEMP i,TEMP k,Lbody,Lend)
LABEL(Lbody)
code for body (here square and print)
MOVE(TEMP i,BINOP(+,TEMP i,1))
JUMP(Lstart)
LABEL(Lend)
\end{verbatim}
Might be expressed as trees.
\end{modelanswer}

\end{subsubquestions}

\end{subquestions}

\newpage

\question

\begin{subquestions}

\subquestion
The following regular expression recognises certain strings consisting of the
letters $a$, $b$ and $c$:
\[
a((ab)|c)*c
\]
Indicate which of these five strings are recognised by the above regular expression:
\begin{quote}
$acc$, $abac$, $ac$, $ababababacac$, $aabcc$
\end{quote}
Also, show three more strings that are recognised by the above expression.
Finally, show two more strings consisting of 
the letters $a$, $b$ and $c$ that are \emph{not} 
recognised by the above regular expression.
\marks{30}

\begin{modelanswer}
Yes, No, Yes, No, Yes. 3 marks each. 
Five further strings, 3 marks each.
\end{modelanswer}


\subquestion
Explain what it means for a context-free grammar to be ambiguous. 
Write down an ambiguous grammar and show why it is ambiguous.
\marks{20}

\begin{modelanswer}
Two parse trees, two derivations, for same sentence. 5 marks.
5 marks for a grammar. 10 for the two trees/derivations.
\end{modelanswer}

\subquestion

Consider the following grammar:
\begin{quote}
\begin{tabbing}
stmtxxx\=$\rightarrow$xxx\=if\kill
\it
E \> $\rightarrow$ \> {\it E\/} {\it B\/} {\it E} \\
\it
E    \> $\rightarrow$ \> \verb!num! \\
\it
E    \> $\rightarrow$ \> ( {\it E\/} ) \\
\it
B     \> $\rightarrow$ \> \verb!+! \\
\it
B     \> $\rightarrow$ \> \verb!-! \\
\it
B     \> $\rightarrow$ \> \verb!*! \\
\it
B     \> $\rightarrow$ \> \verb!/! \\
\end{tabbing}
\end{quote}
\begin{subsubquestions}
\subsubquestion
Explain why this grammar is not suitable to form the basis for a 
recursive descent parser.
\marks{10}

\begin{modelanswer}
Because it's left-recursive and so the usual way of writing the 
procedures leads to immediate recursive call.
\end{modelanswer}

\subsubquestion
Rewrite the rules to obtain an equivalent grammar which can
be used as the basis for a recursive descent parser.
\marks{20}
\end{subsubquestions}

\begin{modelanswer}
\begin{verbatim}
E -> num E'
E -> (E) E'
E' -> B E E'
E' -> empty
\end{verbatim}
\end{modelanswer}

\subquestion
Consider the following Java method:
\begin{verbatim}
1  class A {
2   String a; String c;
3   public void f(int b, int c) {
4     System.out.println(b+c);
5     String d = "hi";
6     int a = b;
7     System.out.println(a); System.out.println(b);
8     System.out.println(c); System.out.println(d);
9   }
10 }
\end{verbatim}
Given an initial environment $\sigma_0$, 
derive the type binding environments for the method at each
use of an identifier and indicate where type lookups will occur.
\marks{20}

\begin{modelanswer}
0  $\sigma_0$ is starting environment\\
2  $\sigma_1 = \sigma_0 + \{a\rightarrow string,c\rightarrow string\}$\\
3  $\sigma_2 = \sigma_1 + \{b\rightarrow int,c\rightarrow int\}$ (overrides instance c)\\
4  lookup ids b, c  in $\sigma_2$\\
5  $\sigma_3 = \sigma_2 + \{d\rightarrow string\}$ \\
6  lookup id b, then $\sigma_4 = \sigma_3 + \{a\rightarrow int\}$ (overrides instance a)\\
7  lookup a, b  in $\sigma_4$\\
8  lookup c, b in $\sigma_4$\\
9  discard $\sigma_4$ revert to $\sigma_1$\\
10 discard $\sigma_1$ revert to $\sigma_0$
\end{modelanswer}


\end{subquestions}

\newpage

\question


\begin{subquestions}

\subquestion
Choose a programming language you know well and describe
how run-time storage is organised and managed during program 
execution.
Clearly associate any storage structures you
mention with the implementation of particular 
language features.
\marks{25}

\begin{modelanswer}
Language features: procedure/method call and parameter passing,
lexical scoping, dynamic storage allocation and deallocation,
including object creation. Name stack, heap, why different.
More marks for good exposition of example language
(eg Java object creation, method call). Operation of
stack and heap (with mention of garbage collection).
\end{modelanswer}


\subquestion
What is a stack frame?
Outline a typical layout for a stack frame and describe 
each element of a frame 
and how it is pushed to the stack during
program execution. 
%Explain the term {\em static link\/} and explain why a C or Java
%implementation does not include static links in its stack frames.
Comment on how local variables, arguments and non-local variables
are addressed by the code generated for a procedure or method. 
\marks{25}

\begin{modelanswer}
A structure to store info local to a proc invocation.
Args, static link, locals, return address, regs/temp space.
Code generated for a proc/func doing the pushing.
Locals, args: offsets from SP or FP; non-locals: follow static links
(more for description of static link).
\end{modelanswer}

\subquestion
Explain why registers might be used for parameter passing and
suggest situations where passing in registers is particularly
appropriate. 
Outline situations where it is necessary for the code generated for a 
procedure or method to write registers to the stack.
\marks{30}

\begin{modelanswer}
Efficiency; particularly appropriate when leaf procs, interproc
reg alloc, dead variables, reg windows (but\ldots).
Reg saves: when address is taken,
when call-by-ref,
when accessed by inner nesting,
value too big,
an array,
convention of save for partic reg prior to call,
spilling in exp evaluation,
saving a reg window.
\end{modelanswer}

\subquestion
Explain the difference between \emph{caller-save\/} and \emph{callee-save\/}
registers. Why might caller-save registers sometimes not be saved?
\end{subquestions}
\marks{20}

\begin{modelanswer}
Caller-save if code for caller of a func saves and restores the reg value
around a func call. Callee save if code for a func does it. Might not be
saved when analysis shows that a value of some var will not be needed following
the call - put in caller-save reg but NOT save it ahead of call.
\end{modelanswer}

\end{questions}

\end{document}

