% Herewith two questions for Principles and Tools

\question

\newenvironment{bnf}{
    \begin{tabbing}
    XXXXXX \= XXXX \= XXXXXXXXXXXXXXXXXXXXXXXXXXXXXXXXXX\kill}{
    \end{tabbing}
    }

\newcommand{\nt}[1]{{\it #1\/}}
\newcommand{\Or}[1]{\ \ $|$ \ \ #1}
\newcommand{\Rule}[2]{{\it #1} \> $\rightarrow$ \> #2}


The following is an extract from the BNF syntax for
a programming language:
\begin{bnf}
\Rule{\nt{stms}}{\nt{stm stms}\Or{\nt{stm}}} \\
\Rule{\nt{stm}}{\nt{if}\Or{\nt{assign}}} \\
\Rule{\nt{assign}}{\nt{var} \verb":=" \nt{aexp} \verb";"} \\
\Rule{\nt{if}}{\verb"if" \nt{bexp} \verb"then" \nt{stms} \verb"else" \nt{stms} \verb"endif ;"}\\
\Rule{\nt{bexp}}{\nt{aexp} \verb"=" \nt{aexp}\Or{\nt{aexp} \verb"<=" \nt{aexp}}}\\
\Rule{\nt{aexp}}{\nt{var}\Or{\nt{num}}\Or{\nt{aexp} \verb"+" \nt{aexp}}
        \Or{\nt{aexp} \verb"-" \nt{aexp}} \Or{\nt{aexp} \verb"*" \nt{aexp}}}
\end{bnf}
where \nt{var} and \nt{num} have obvious interpretations.
\begin{subquestions}
\subsubquestion
Show that this is an ambiguous grammar by constructing different
concrete syntax trees for the assignment:\marks{4} 
\begin{verbatim}
       t := m + 60 * h
\end{verbatim}
%\subquestion
%Modify the grammar so that the \verb"+" and \verb"-" operators
%have equal precedence lower than that of \verb"*".
%\marks{4}
\subquestion
The meaning of an \nt{if} is as follows:
\begin{itemize}
\item On entry to the \nt{if}, the \nt{bexp} is 
        evaluated. 
\item If the value of \nt{bexp} is true,
        the \nt{stms} following the \verb"then" are
        executed and then control transfers to the statement following
        the \verb"endif". 
\item Otherwise, the \nt{stms} following the \verb"else" are
        executed,
        and then control tranfers to the statement following
        the \verb"endif".
\end{itemize}
\begin{subsubquestions}
\subsubquestion
	Outline the structure of the
        code that might be generated in translation of
        an \nt{if} with a typical \nt{bexp} (\verb"a+1 <= b-1", say). 
	Assume code generation for a simple
        stack or register machine, and briefly explain the
	instructions you use in your outline.
\marks{4}
\subsubquestion
        Construct a syntax-directed translation for 
        \nt{if} and \nt{bexp} which produces such code.
        Describe, in detail, any attributes you
        introduce, and make clear any assumptions you make about 
        the attributes and code generated for the
        expressions \nt{aexp}, but you need not produce translations for
	\nt{aexp}.
\marks{10}
\end{subsubquestions}

\subquestion
Outline the Sethi-Ullman numbering algorithm and
note two potential uses for it.  Assuming two registers
are required for loading a variable, and one for loading a literal number,
draw the labelled
tree produced by your algorithm for the expression:
\marks{7}
\begin{verbatim}
       (a - 3) + ((c + 4) + (e * f))
\end{verbatim}

\end{subquestions}

\question

\begin{subquestions}
\subquestion
Draw a diagram of the usual run time memory
subdivision (memory model) for execution of
programs compiled from languages such as
Pascal and C.  Briefly describe the function of each
of the subdivisions (areas), and discuss
the programming language features which influence
the organisation and functioning of run time storage.
\marks{7}

\subquestion
Sketch a possible structure for a stack activation
record (stack frame) and briefly explain the purpose of
each component. Comment on how local variables,
parameters and non-local variables
are addressed by the code generated for a
procedure when a run-time stack is in use,
and explain the need for a register `spill' area
in some records.
\marks{9}

\subquestion

        Explain what is meant by the parameter passing mechanisms usually
        known as
                \begin{enumerate}
                \item call-by-value
                \item call-by-reference
                \item call-by-value-result
%                \item call-by-name
                \end{enumerate}
        Deduce for each mechanism what is printed by the 
        program shown below, assuming that all parameters are
	passed with the same mechanism.
        In each case briefly justify your answer.
\begin{verbatim}
       MODULE prettypass; 
       
       IMPORT InOut;

       VAR a, b: INTEGER;
 
       PROCEDURE pass (Param x,y,z: INTEGER);
       BEGIN
            y := y + 1;
            z := z + x
       END pass;
               
       BEGIN
            a := 2;
            b := 3;
            pass (a+b, a, a);
            InOut.WriteInt (a)  /* Prints value of `a'. */
       END.
\end{verbatim}                   
Although the program is written in a Modula-2-like syntax, it
is not a valid Modula-2 program.
\marks{9}

\end{subquestions}
