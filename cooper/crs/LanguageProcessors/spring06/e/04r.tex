%\documentstyle[cityexam,11pt]{article}
\documentclass[11pt]{bareexam}
\usepackage{semantic}
%\withmodelanswers
\begin{document}
%\examnumber{???}
%\begin{preamble}
%\degrees{???}

%\part{II}
%\title{Language Processors}
%\examdate{???}
%\examtime{???}
%\rubric{Answer {\sl TWO} questions \\ All questions carry equal marks}
%\end{preamble}
%\externals{???}
\internals{\ }

\reservestyle{\nonterm}{\textit}
\nonterm{Statement,Exp,dec,decs,vardec,fundec,id,exp,tyfields,type-id,fundecs,expseq,op}
\reservestyle{\keyword}{\textbf}
\keyword{var,function,let,in,end,if,then,else,while,do,for,break,to,switch,of,case,default}


\begin{questions}

\question

\setcounter{page}{2}

\begin{subquestions}

%\subquestion
%Choose a programming language you know well and describe
%how run-time storage is organised and managed during program 
%execution.
%Clearly associate any storage structures you
%mention with the implementation of particular 
%language features.
%\marks{25}
%
%\begin{modelanswer}
%Language features: procedure/method call and parameter passing,
%lexical scoping, dynamic storage allocation and deallocation,
%including object creation. Name stack, heap, why different.
%More marks for good exposition of example language
%(eg Java object creation, method call). Operation of
%stack and heap (with mention of garbage collection).
%\end{modelanswer}


\subquestion
%What is a stack frame?
%Outline a typical layout for a stack frame and describe 
%each element of a frame 
%and how it is pushed to the stack during
%program execution. 
%Explain the term {\em static link\/} and explain why a C or Java
%implementation does not include static links in its stack frames.
What programming language feature leads to the need for an
implementation model that includes stack frames? Explain your answer.
Explain in detail how a stack frame is pushed to the stack, and 
removed from the stack, during program execution. 
\marks{25}

\begin{modelanswer}
Procedure or method calls, recursion, need for separate storage space for
parameters and locals.
Code generated for a proc/func does the pushing/popping.
\begin{verbatim}
caller g(...) calls callee f(a1,...,an)
calling code in g puts arguments to f at end of g frame
stores return address
referenced through SP, incrementing SP
on entry to f, SP points to first argument g passes to f
old SP becomes current frame pointer FP
f then allocates frame by setting SP=(SP - framesize)
old SP becomes current frame pointer FP
f then initialises locals
on exit from f : SP = FP, removing frame
jumps to return address
\end{verbatim}
\end{modelanswer}

\subquestion
Comment on how local variables, arguments and non-local variables
are addressed in a stack frame by 
the code generated for a procedure or method. 
\marks{15}

\begin{modelanswer}
Locals, args: offsets from SP or FP; non-locals: follow static links
(more for description of static link).
\end{modelanswer}

\subsubquestion
Suppose  that a compiler translates a MiniJava-like language
to an intermediate representation (for example IR trees) 
that will include 
the calculations required to address variables in stack frames.
Draw or write down
the intermediate representation
required to access a local variable declared in a method.
Explain your answer.
\marks{25}

\begin{modelanswer}
MEM(BINOP(PLUS,TEMP fp, CONST k)) where k is offset of var in frame, fp the
register holding the framepointer. Has to compute place in frame.
\end{modelanswer}


\subquestion
Explain why registers might be used for parameter passing and
suggest situations where passing in registers is particularly
appropriate. 
Outline situations where it is necessary for the code generated for a 
procedure or method to write registers to the stack.
\marks{20}

\begin{modelanswer}
Efficiency; particularly appropriate when leaf procs, interproc
reg alloc, dead variables, reg windows (but\ldots).
Reg saves: when address is taken,
when call-by-ref,
when accessed by inner nesting,
value too big,
an array,
convention of save for partic reg prior to call,
spilling in exp evaluation,
saving a reg window.
\end{modelanswer}

\subquestion
Explain the difference between \emph{caller-save\/} and \emph{callee-save\/}
registers. Why might caller-save registers sometimes not be saved?
\marks{15}

\begin{modelanswer}
Caller-save if code for caller of a func saves and restores the reg value
around a func call. Callee save if code for a func does it. Might not be
saved when analysis shows that a value of some var will not be needed following
the call - put in caller-save reg but NOT save it ahead of call.
\end{modelanswer}

\end{subquestions}

\newpage

\question

The reference manual for a MiniJava-like programming language contains
the following grammar fragment for an expression involving the \verb+?:+ (short-cut if-else) operator: 

\[
\<Exp> \; -> \; \<Exp> \;\; ? \;\; \<Exp> \;\; : \;\; \<Exp>
\]

We will call this expression the short-cut if-else expression.
This expression has the same meaning and type-checks as expressions
involving the short-cut if-else operator in Java -- 
\verb+op1?op2:op3+ returns \verb+op2+ if \verb+op1+ is true,
and returns \verb+op3+ if \verb+op1+ is false.


\begin{subquestions}

\subquestion
Sketch a possible abstract syntax for the short-cut if-else expression.
\marks{15}

\begin{modelanswer}
\begin{verbatim}
SCIFELSE Exp Exp Exp
(be flexible about how this is expressed - it might be Java and have types)
\end{verbatim}
\end{modelanswer}

\subquestion
Show how semantic actions in a grammar for a parser-generator such as JavaCC
can be used to produce abstract syntax trees for the short-cut 
if-else expression
\marks{30}

\begin{modelanswer}
\begin{verbatim}
Exp ShortCutIfElseExpression() :
{ Exp e1,e2,e3; }
{
  e1=PrimaryExpression() "?" e2=PrimaryExpression() ":"
      e3 = PrimaryExpression()
  { return new ShortCutIfElse(e1,e2,e3); }
}
\end{verbatim}
\end{modelanswer}

\subquestion
Informally describe an appropriate typecheck for the short-cut 
if-else expression.
\marks{20}

\begin{modelanswer}
e1 should be boolean, e2 same type as e3 according to oo rules.
\end{modelanswer}

\subquestion

Suppose a compiler for a MiniJava-like language that includes
a short-cut if-else expression translates all statements and expressions
into intermediate code (eg intermediate representation (IR) trees).
Outline the intermediate code that might be generated
in translation of the short-cut if-else expression. 
You may wish to use a simple
example to explain your translation, eg:
\begin{verbatim}
c = a < b ? a + b : a - b ;
\end{verbatim}
You can assume that the expression tree for any variable \verb"v" is
simply \verb"TEMP v". 
\marks{35}

\begin{modelanswer}
\begin{verbatim}
SEQ(SEQ(CJUMP(LT, TEMP a, TEMP b, L0, L1),
        SEQ(SEQ(SEQ(LABEL L0,
                  MOVE(TEMP t, BINOP(PLUS, TEMP a, TEMP b))),
                JUMP(NAME L2)),
            SEQ(SEQ(LABEL L1,
                  MOVE(TEMP t, BINOP(MINUS, TEMP a, TEMP b),
                JUMP(NAME L2)))),
     (SEQ (LABEL L2, MOVE(TEMP c, TEMP t)))
        )
))

(might be a longer translation where a < b is evaluated to 
0 or 1 and then checked, might be just a linear translation too)

\end{verbatim}
Might be expressed as trees.
\end{modelanswer}

\end{subquestions}

\newpage

\question

\begin{subquestions}

\subquestion
The following regular expression recognises certain strings consisting of the
letters $a$, $b$ and $c$:
\[
a(b|(bc))\!*c*
\]
Indicate which of these five strings are recognised by the above regular expression:
\begin{quote}
$acc$, $abac$, $a$, $abcbcbccc$, $abbbccbc$
\end{quote}
Also, show three more strings that are recognised by the above expression.
Finally, show two more strings consisting of 
the letters $a$, $b$ and $c$ that are \emph{not} 
recognised by the above regular expression.
\marks{25}

\begin{modelanswer}
Yes, No, Yes, Yes, Yes. 3 marks each. 
Five further strings, 2 marks each.
\end{modelanswer}


\subquestion
Consider the following grammar, which we will call {\it E\/}:
\begin{quote}
\begin{tabbing}
stmtxxx\=$\rightarrow$xxx\=if\kill
\it
E \> $\rightarrow$ \> {\it E\/} \verb!+! {\it E} \\
\it
E \> $\rightarrow$ \> {\it E\/} \verb!-! {\it E} \\
\it
E \> $\rightarrow$ \> ( {\it E\/} ) \\
\it
E \> $\rightarrow$ \> \verb!num!
\end{tabbing}
\end{quote}

\begin{subsubquestions}
\subsubquestion
Explain what it means for a context-free grammar to be ambiguous. 
Show that grammar {\it E\/} is ambiguous.
\marks{20}

\begin{modelanswer}
Two parse trees, two derivations, for same sentence. 5 marks.
15 for the two trees/derivations.
\end{modelanswer}

\subsubquestion
Explain why grammar {\it E\/} is not suitable to form the basis for a 
recursive descent parser.
\marks{10}

\begin{modelanswer}
Because it's left-recursive and so the usual way of writing the 
procedures leads to immediate recursive call.
\end{modelanswer}

\subsubquestion
Rewrite the rules to obtain an equivalent grammar which can
be used as the basis for a recursive descent parser.
\marks{25}

\begin{modelanswer}
\begin{verbatim}
E -> num E'
E -> (E) E'
E' -> + E E'
E' -> - E E'
E' -> empty
\end{verbatim}
\end{modelanswer}

\end{subsubquestions}

\subquestion
Consider the following Java method:
\begin{verbatim}
1  class A {
2   String a; int c;
3   public void f(int b, String c) {
4     System.out.println(c);
5     int d = 3;
6     int a = b;
7     System.out.println(a+d); System.out.println(b);
8     System.out.println(c); System.out.println(d);
9   }
10 }
\end{verbatim}
Given an initial environment $\sigma_0$, 
derive the type binding environments for the method at each
use of an identifier and indicate where type lookups will occur.
\marks{20}

\begin{modelanswer}
0  $\sigma_0$ is starting environment\\
2  $\sigma_1 = \sigma_0 + \{a\rightarrow string,c\rightarrow int\}$\\
3  $\sigma_2 = \sigma_1 + \{b\rightarrow int,c\rightarrow String\}$ (overrides instance c)\\
4  lookup id c  in $\sigma_2$\\
5  $\sigma_3 = \sigma_2 + \{d\rightarrow int\}$ \\
6  lookup id b, then $\sigma_4 = \sigma_3 + \{a\rightarrow int\}$ (overrides instance a)\\
7  lookup a, d, b  in $\sigma_4$\\
8  lookup c, b in $\sigma_4$\\
9  discard $\sigma_4$ revert to $\sigma_1$\\
10 discard $\sigma_1$ revert to $\sigma_0$
\end{modelanswer}


\end{subquestions}

\end{questions}

\end{document}
