%From seb Wed May 11 17:13:30 1994
%%Date: Wed, 11 May 1994 17:13:27 +0100 (BST)
%From: Sebastian Hunt <seb>
%Subject: 
%To: David Bolton <daveb>
%Message-Id: <Pine.3.89.9405111714.A12954-0100000@barney.cs.city.ac.uk>
%Mime-Version: 1.0
%Content-Type: TEXT/PLAIN; charset=US-ASCII
%Status: RO

\documentstyle[/homes/daveb/doc/exam_style/cityexam,11pt]{article}
\begin{document}
\examnumber{00.00}
\begin{preamble}
\degrees{EC2 retread}
\part{II}
\title{Programming Systems and Principles of Programming Languages}
\examdate{XXX, 00 XXX, 1994}
\examtime{00 to 00}
\rubric{Two from section A, two from B}
\end{preamble}
\externals{XXX \\ XXX}
\internals{S Hunt \\ DJ Bolton}

\examsection{A}

% Herewith three questions for Language Processors modified for PS, EC student DB 12.5.94

\begin{questions}

\question

\newenvironment{bnf}{
    \begin{tabbing}
    XXXXXX \= XXXX \= XXXXXXXXXXXXXXXXXXXXXXXXXXXXXXXXXX\kill}{
    \end{tabbing}
    }

\newcommand{\nt}[1]{{\it #1\/}}
\newcommand{\Or}[1]{\ \ $|$ \ \ #1}
\newcommand{\Rule}[2]{{\it #1} \> $\rightarrow$ \> #2}


The following is an extract from the BNF syntax for
a programming language:
\begin{bnf}
\Rule{\nt{stms}}{\nt{stm stms}\Or{\nt{stm}}} \\
\Rule{\nt{stm}}{\nt{loop}\Or{\nt{assign}}} \\
\Rule{\nt{assign}}{\nt{id} \verb":=" \nt{exp} \verb";"} \\
\Rule{\nt{loop}}{\verb"for" \nt{id} \verb"=" \nt{exp$_1$} \verb"to" \nt{exp$_2$}
 \verb"do" \nt{stms} \verb"endfor ;"} \\
\Rule{\nt{exp}}{\nt{id}\Or{\nt{integer}}\Or{\nt{exp} \verb"+" \nt{exp}}
        \Or{\nt{exp} \verb"-" \nt{exp}} \Or{\nt{exp} \verb"*" \nt{exp}}}
\end{bnf}
where \nt{id} and \nt{integer} have obvious interpretations.
\begin{subquestions}
\subsubquestion
Show that this is an ambiguous grammar by constructing different
syntax trees for the assignment:\marks{6}
\begin{verbatim}
       t := m + 60 * h
\end{verbatim}
%\subquestion
%Modify the grammar so that the \verb"+" and \verb"-" operators
%have equal precedence lower than that of \verb"*".
%\marks{4}
\subquestion
The meaning of a \nt{loop} is as follows:
\begin{itemize}
\item On entry to the loop, arithmetic expressions \nt{exp$_1$} and \nt{exp$_2$}
        are evaluated. The value of \nt{exp$_1$} is the {\em lower
        bound}, and is assigned to the identifier \nt{id}.
        The value of \nt{exp$_2$} is the {\em upper bound}.
        Neither may be changed after entry to the loop by 
        execution of \nt{stms}.
\item If the value of \nt{exp$_2$} is less than \nt{id},
        the loop body \nt{stms} is
        not executed and control transfers to the statement following
        the \verb"endfor" (leaving the loop).
\item Otherwise, the loop body \nt{stms} is
        executed,  
        then the value of \nt{id} is incremented by 1,
        and control tranfers back to the start of the loop.
\end{itemize}
\begin{subsubquestions}
\subsubquestion
	Outline and describe the structure of the three-address
        code that might be generated in translation of
        a \nt{loop}.  
        Briefly explain the
	three-address code you use in your outline.
\marks{6}
\subsubquestion
        Construct a syntax-directed translation for 
        \nt{loop} which produces such code.
        Describe, in detail, any attributes you
        introduce, and make clear any assumptions you make about 
        the attributes and code generated for the
        expressions \nt{exp}, but do not produce translations for
	\nt{exp}.
\marks{13}
\end{subsubquestions}
\end{subquestions}

\question

\begin{subquestions}
\subquestion
Draw a diagram of the usual run time memory
subdivision (memory model) for execution of
programs compiled from languages such as
Pascal and C.  Briefly describe the function of each
of the subdivisions (areas), and discuss
the programming language features which influence
the organisation and functioning of run time storage.
\marks{8}

\subquestion
Sketch a possible structure for a stack activation
record (stack frame) and briefly explain the purpose of
each component. Comment on how local variables,
parameters and non-local variables
are addressed by the code generated for a
procedure when a run-time stack is in use,
and explain the need for a register `spill' area
in some records.
\marks{10}

\subquestion

Using examples describe the structure and function of the
assembly code
that must be produced by a code generator to manage a run-time
stack at the time of procedure call and return.
\marks{7}

\end{subquestions}


 
\question

\begin{subquestions}

\subquestion
Outline the Sethi-Ullman numbering algorithm and 
describe potential uses for it.  Assuming two registers
are required for loading a variable, and one for loading a literal number, 
draw the labelled
tree produced by your algorithm for the expression:
\marks{9}
\begin{verbatim}
       (a - 3) + ((c + 4) + (e * f))
\end{verbatim}

\subquestion
        Explain what is meant by the parameter passing mechanisms usually
        known as
                \begin{enumerate}
                \item call-by-value
                \item call-by-reference
                \item call-by-value-result
                \item call-by-name
                \end{enumerate}
        Deduce for each mechanism what is printed by the 
        program shown below, assuming that all parameters are
	passed with the same mechanism.
        In each case briefly justify your answer.
\begin{verbatim}
       MODULE prettypass; 
       IMPORT InOut;

       VAR a, b: INTEGER;
 
       PROCEDURE pass (Param x,y,z: INTEGER);
       BEGIN
            y := y + 1;
            z := z + x
       END pass;
               
       BEGIN
            a := 2;
            b := 3;
            pass (a+b, a, a);
            InOut.WriteInt (a)
       END.
\end{verbatim}                   
Although the program is written in a Modula-2-like syntax, it
is not a valid Modula-2 program.  The procedure \verb"InOut.WriteInt"
simply writes the value of its argument to the standard output and
\verb"a" and \verb"b" are global variables.
\marks{16}

\end{subquestions}




% Seb's bit here. 12.5.94

\examsection{B}

	\question

	\begin{subquestions}

		\subquestion 

What is meant by the term {\em aliasing}? Using examples,
describe two ways in which aliasing can occur in Pascal programs
{\em without} using pointers.
	\marks{8}

		\subquestion

What is meant by the term {\em block-structure}?
Explain why both an environment and a store are used to describe the semantics of
a block-structured language. Explain what is meant by
the terms {\em scope} and {\em hole in scope}.
	\marks{8}

		\subquestion

Explain the distinction between {\em static binding} and {\em dynamic binding}
in a block-structured language with user defined procedures.
Use an example to show how the choice between static and dynamic binding can
determine the effect of a procedure invocation.

What are the advantages of static binding?
	\marks{9}

	\end{subquestions}

	\question

Part of the syntax of a simple language of commands is as follows:

	\begin{quote}
	\begin{tabular}{lcl}
	 C $\in$ Command	&$::=$& {\bf skip}
				$\vert$ I $:=$ E
				$\vert$ C ; C \\
%%				$\vert$ {\bf if} B {\bf then} C {\bf else} C\\
%%	 B $\in$ Bexp		&$::=$& E $<$ E
%%				$\vert$ E $=$ E
%%				$\vert$ B {\bf or} B
%%				$\vert$ B {\bf and} B
%%				$\vert$ {\bf not} B
	 E $\in$ Aexp		&$::=$& I
				$\vert$ N
				$\vert$ E $+$ E
				$\vert$ E $*$ E
				$\vert$ E $-$ E	\\
	 I $\in$ Identifier \\
	 N $\in$ Numeral
        \end{tabular}
        \end{quote}

For this simple language, stores can be modelled by the semantic domain
        \begin{quote}
         s $\in$ Store $=$ Identifier $\rightarrow$ ${\cal Z}$
        \end{quote}
where ${\cal Z}$ is the set of integers.

	\begin{subquestions}

                \subquestion

Assuming the existence of a map
{\bf N} : Numeral $\rightarrow$ ${\cal Z}$,
give equations defining the semantic maps
        \begin{quote}
         {\bf E} : Aexp $\rightarrow$ Store $\rightarrow$ ${\cal Z}$ \\
         {\bf C} : Command $\rightarrow$ Store $\rightarrow$ Store
        \end{quote}
	\marks{10}

		\subquestion

\begin{enumerate}
  \item Suggest a grammar for boolean expressions, including tests for
	equality and inequality of arithmetic expressions, boolean
	conjunction, disjunction and negation.
	\marks{5}

  \item	Write down the type of an appropriate semantic map for boolean
	expressions and give equations defining this map.
	\marks{5}
\end{enumerate}

		\subquestion

Extend the grammar of commands to include the if-then-else construct.
Add one or more equations to the definition of the semantic map {\bf C}
to cater for the if-then-else construct.
	\marks{5}

	\end{subquestions}

	\question

The higher-order functions \verb"map" and \verb"foldr" are defined as follows:
	\begin{quote}
	\begin{tabular}{l}
	 \verb"map f []     = []" \\
	 \verb"map f (x:xs) = f x : map f xs" \\
	 \\
	 \verb"foldr f b []     = b" \\
	 \verb"foldr f b (x:xs) = f x (foldr f b xs)"
	\end{tabular}
	\end{quote}

	\begin{subquestions}

		\subquestion 

Define the term {\em higher-order function}.
		\marks{3}

		\subquestion 

The functions {\it map} and {\it foldr} have polymorphic types. Use these
functions as examples to explain what is meant by {\em polymorphic function},
and write down the types of both {\it map} and {\it foldr}.
		\marks{8}

		\subquestion 

Explain what is meant by the term {\em shallow type}. Write down a
{\em non}-shallow type for the following function:
\begin{quote}
 \verb"strange f = f 3 + f True"
\end{quote}
		\marks{6}

		\subquestion 

The function
\begin{quote}
\verb"and : (list bool) -> bool"
\end{quote}
returns \verb"False" for lists
which contain \verb"False", and returns \verb"True" for all other lists.
Define \verb"and" in two different ways:
\begin{enumerate}
 \item Using explicit recursion.
 \item Using \verb"foldr".
\end{enumerate}
		\marks{8}

	\end{subquestions}

\end{questions}

\end{document}




