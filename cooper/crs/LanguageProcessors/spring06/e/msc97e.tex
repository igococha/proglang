\documentstyle[/homes/daveb/doc/exam_style/cityexam,11pt]{article}
\withmodelanswers               % Comment this out to produce bare paper
\begin{document}
\examnumber{00.00}
\begin{preamble}
\degrees{EC2 retread}
\part{II}
\title{Programming Systems and Principles of Programming Languages}
\examdate{XXX, 00 XXX, 1994}
\examtime{00 to 00}
\rubric{Two from section A, two from B}
\end{preamble}
\externals{XXX \\ XXX}
\internals{S Hunt \\ DJ Bolton}

\begin{questions}

% Herewith two questions for Principles and Tools

\question

\newenvironment{bnf}{
    \begin{tabbing}
    XXXXXXX \= XXXX \= XXXXXXXXXXXXXXXXXXXXXXXXXXXXXXXXXX\kill}{
    \end{tabbing}
    }

\newcommand{\nt}[1]{{\it #1\/}}
\newcommand{\Or}[1]{\ \ $|$ \ \ #1}
\newcommand{\Rule}[2]{{\it #1} \> $\rightarrow$ \> #2}

\begin{subquestions}
\subquestion
If $A$ is a set of characters selected from an
alphabet of characters $V$, the shorthand notation {\it Not$(A)$} denotes 
$(V - A)$; that is all characters in $V$ not included in $A$.
Specify the following patterns using regular expressions; 
the {\it Not\/} shorthand may be useful for some of the patterns:
\begin{subsubquestions}
\subsubquestion
A comment that begins with \verb"--" and ends with the \verb"Eol" 
(end of line) character (use \verb"Eol" to denote the end of line 
character).
\marks{5}

\begin{modelanswer}
\begin{verbatim}
--(Not(Eol))*Eol
\end{verbatim}
\end{modelanswer}
\subsubquestion
An identifier, composed of letters, digits, and underscores, that
begins with a letter, ends with a letter or digit, and contains
no consecutive underscores.
\marks{10}
\begin{modelanswer}
\begin{verbatim}
L((L|D)*(_(L|D)+)*      L=[a-zA-Z] D=[0-9]
\end{verbatim}
\end{modelanswer}
\subsubquestion
A comment delimited by \verb"##" markers, which allows {\em single\/}
\verb"#"'s within the comment body.
\marks{15}
\begin{modelanswer}
\begin{verbatim}
##((#|e)Not(#))*##    e=empty
\end{verbatim}
\end{modelanswer}
\end{subsubquestions}

\subquestion
The following is an extract from a possible BNF syntax for
a subset of an Eiffel-like programming language: 
\begin{bnf}
\Rule{\nt{Compound\ }}{\nt{Instruction} \verb";" \nt{Compound}\Or{\nt{Instruction}}} \\
\Rule{\nt{Instruction\ }}{\nt{Conditional}\Or{\ldots}} \\
\Rule{\nt{Conditional}}{\verb"if" \nt{BoolExp} \verb"then" \nt{Compound$_1$} \verb"else" \nt{Compound$_2$} \verb"end"}\\
\Rule{\nt{IntExp}}{\nt{IntVar}\Or{\nt{IntConstant}}\Or{\nt{IntExp} \verb"+" \nt{IntExp}}
        \Or{\nt{IntExp} \verb"*" \nt{IntExp}}}\\
\Rule{\nt{BoolExp}}{\nt{IntExp} \verb"=" \nt{IntExp}\Or{\nt{IntExp} \verb"<=" \nt{IntExp}}}
\end{bnf}
where \nt{IntVar} is a variable attribute of type INTEGER, \nt{IntConstant}
is an INTEGER type constant (eg 3).

The meaning of a \nt{Conditional} is as follows:
\begin{itemize}
\item On entry to the instruction, the \nt{BoolExp} is 
	evaluated. 
\item If the value of the \nt{BoolExp} is true, the effect of
	the \nt{Conditional} is the effect of \nt{Compound$_1$}.
\item Otherwise, when the \nt{BoolExp} evaluates to false,
	the effect of the \nt{Conditional} is the effect of
	of \nt{Compound$_2$}.
\end{itemize}
\begin{subsubquestions}
%\subquestion
%Show that the grammar rules for \nt{IntExp} are ambiguous.
%\marks{20}
\subsubquestion
Write down an abstract syntax for \nt{Conditional}, 
\nt{BoolExp} and \nt{IntExp}.
\marks{10}
\begin{modelanswer}
\begin{verbatim}
Conditional -> COND BoolExp Compound Compound
BoolExp     -> BEQ IntExp IntExp | BLE IntExp IntExp
IntExp      -> INTVAR name | INTCONST int | INTPLUS IntExp IntExp |
                 INTMULT IntExp IntExp
\end{verbatim}
\end{modelanswer}
\subsubquestion
	Outline the structure of the
        code that might be generated in translation of
        a \nt{Conditional} with an example 
	\nt{BoolExp} (\verb"a+1 <= b*2", say). 
	Assume code generation for a simple
        stack machine, and briefly explain the
	instructions you use in your outline.
\marks{20}
\begin{modelanswer}
\begin{verbatim}
code for RHS IntExp, result on stack top
code for LHS IntExp, result on stack top
LE                   top <= top-1
JUMPFALSE LABEL1
code for Compound1
JUMP LABEL2
LABEL1:
code for Compound2
LABEL2:
\end{verbatim}
Plus explanation.
\end{modelanswer}

\subsubquestion
        Construct a syntax-directed translation for 
        \nt{Conditional} and \nt{BoolExp} which produces such code.
        Describe any attributes you
        introduce, and make clear any assumptions you make about 
        the attributes and code generated for the
        expressions \nt{IntExp}, but you need not produce translations for
	\nt{IntExp}.
\marks{35}
\begin{modelanswer}
\begin{verbatim}
Conditional -> COND BoolExp { 
                         elsepart := newlabel()
                         gen (JUMPFALSE elsepart) }
               Compound {
                         end := newlabel()
                         gen (JUMP end)
                         gen (LABEL elsepart) }
               Compound {
                         gen (LABEL end) }
BoolExp     -> BEQ IntExp IntExp { gen (EQ) }
            |  BLE IntExp IntExp { gen (LE) }
\end{verbatim}
Plus explanation for attrs, results of IntExp translation.
\end{modelanswer}
\end{subsubquestions}

\end{subquestions}

\question

\begin{subquestions}
\subquestion
%Draw a diagram of the usual run time memory
%subdivision (memory model) for execution of
%programs compiled from languages such as
%Pascal and C.  
%Briefly describe the function of each
%of the subdivisions (areas), 
%and 
Discuss the programming language features which influence
the organisation and functioning of run time storage,
giving examples from programming languages you
have encountered.  Briefly describe the run time operation of
each storage structure you identify.
\marks{30}
\begin{modelanswer}
Language features: procedure call and parameter passing, 
lexical scoping, dynamic storage allocation and deallocation, 
including object creation. More marks for example languages
(eg Eiffel object creation, Pascal `new' etc). Operation of 
stack and heap (with mention of garbage collection).
\end{modelanswer}

\subquestion
%Explain what is meant by {\em fragmentation} in heaps, 
%and mention some methods for reducing fragmention. 
Define the terms {\em garbage\/} and {\em dangling pointer} 
and explain using simple example program fragments 
how each arises.
%Sketch the simple mark-sweep garbage collection algorithm
%for automatic deallocation of inaccessible heap objects.
\marks{20}
\begin{modelanswer}
Half marks for each.
\end{modelanswer}

\subquestion
Explain what is meant by the argument passing mechanisms usually
        known as
                \begin{enumerate}
                \item call-by-value
                \item call-by-reference
                \item call-by-value-result
%                \item call-by-name
                \end{enumerate}
        Deduce for each mechanism what is the effect of
        the following routine:
\begin{verbatim}
       pass (p,q,r: INTEGER) is do
            q := q + 1;
            r := r + p
       end
\end{verbatim}   
        when the sequence of assigments and a feature call
        shown below is executed, assuming that all arguments are
        passed with the same mechanism, and that \verb"a" and
        \verb"b" are variable attributes of type INTEGER.
        Write down what is printed in each case by the \verb"putint"
        call, and clearly justify your answer by reference to your
        description of the argument passing mechanism.
\begin{verbatim}
       a := 2;
       b := 3;
       pass (a+b,a,a);
       io.putint (a)
\end{verbatim}
Although the program is written in Eiffel syntax, the Eiffel
argument-passing semantics do not apply and you must
instead consider each of the mechanisms mentioned, and assume
that it is legal in a routine to assign to a formal argument.
\marks{30}
\begin{modelanswer}
\begin{verbatim}
cbv explanation -- answer is 2.
cbr explanation -- 
   a := a+1 = 3
   a := a+5 = 8    answer 8
cbvr explanation
   5,2,2 passed
   q := q+1 = 3
   r := r+5 = 7    answer 7 or 3 depending on copy back order
\end{verbatim}
\end{modelanswer}

\subquestion
Explain why the following program does not provide a fair spin lock
for exclusive access to a shared resource:
\begin{verbatim}
acquire:
  claimed[self] := true;
  while claimed[other] do
    begin
      claimed[self] := false;
      while claimed[other] do skip;
      claimed[self] := true
    end

relinquish:
  claimed[self] := false
\end{verbatim}
Correct the fault by introducing the use of a turn variable
(Dekker's algorithm).
\marks{20}
\begin{modelanswer}
Because an instruction by instruction interleaving means neither
ever acquires.
\begin{verbatim}
acquire:
  claimed[self] := true;
  while claimed[other] do
    if turn = other then
      begin
        claimed[self] := false;
        while turn = other do skip;
        claimed[self] := true
    end

relinquish:
  turn := other;
  claimed[self] := false
\end{verbatim}
\end{modelanswer}
\end{subquestions}
\end{questions}

\end{document}

