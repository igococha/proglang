\documentclass{article}

\usepackage{a4wide}
\usepackage[breaklinks=true,
        colorlinks=true,urlcolor=blue,pdfpagemode=None]{hyperref}

\begin{document}
\thispagestyle{empty}

\section*{Language Processors Lab 5}

{\bf Note:} Read this through {\em before\/} logging in.

\medskip\noindent The goal for this
lab is to see the MiniJava typechecker in operation.


\subsection*{MiniJava typechecker}

The MiniJava symbol table builder and typechecker is at
 \verb+/soi/sw/courses/daveb/IN2009/minijava/chap5+.
\verb+README+ files in each directory explain the basic structure
and how to compile and run it.
Move to where you want to do your work and copy it with:
\begin{verbatim}
cp -R /soi/sw/courses/daveb/IN2009/minijava/chap5 .
\end{verbatim}
Compile with:
\begin{verbatim}
module add java soi javacc/3.2
javacc minijava.jj
javac Main.java
\end{verbatim}
and run with:
\begin{verbatim}
java Main filename
\end{verbatim}
where \verb+filename+ is a file containing a MiniJava program (eg one of
those from the \verb+minijava/programs+ directory).
The output is a pretty-printed version of the program, and any error messages
resulting from the typechecking.

The programs I've supplied contain no type errors so you won't
see any error messages from them. You should now create some of
your own test program files that contain simple errors that
should be detected by the symbol table builder and type checker, and try them out.

Then read the slides from Session 5 and the visitor programs and try to 
understand how the symbol table is built and how the typechecker works.


\end{document}

