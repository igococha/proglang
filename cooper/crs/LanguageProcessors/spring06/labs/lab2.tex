\documentclass{article}

\usepackage{a4wide}
\usepackage[breaklinks=true,
        colorlinks=true,urlcolor=blue,pdfpagemode=None]{hyperref}



\begin{document}

\thispagestyle{empty}

\subsection*{Language Processors Lab 2}

{\bf Note:} Read this through {\em before\/} logging in.

\medskip\noindent The goal for this
lab is to see a JavaCC-generated lexical analyser working to
simply print out some recognised tokens, and
then to write some of your own expressions and implement them.

\subsubsection*{A trivial lexical analyser}

Fire up a Unix shell window. 
To get the environment ready type the command:
\begin{verbatim}
   module add java soi javacc/3.2
\end{verbatim}
The module \verb+javacc/3.2+ gives you access to the JavaCC tool that includes
a lexical analyser builder.

I have written a simple JavaCC input file which recognises a small number of
token types.  Move to the directory in which you want to do
your IN2009 work and copy the example directory with the command:
\begin{verbatim}
   cp -R /soi/sw/courses/daveb/IN2009/lextest .
   cd lextest
\end{verbatim}
You can see the JavaCC file with:
\begin{verbatim}
   more LexTest.jj
\end{verbatim}
(or you can edit it with your editor of choice, of course).
Now run JavaCC on the script file with
\begin{verbatim}
   javacc LexTest.jj
\end{verbatim}
This produces a Java program in various files.
This program is the lexical analyser and recognizes the
tokens specified in the \verb+LexTest.jj+ file.
Now compile these Java classes with:
\begin{verbatim}
   javac *.java
\end{verbatim}
And then run the program: 
\begin{verbatim}
   java LexTest
\end{verbatim}
Type in some identifier names and integers and see
what happens.

Now look at the file \verb+LexTest.jj+.
The two forms of token (SKIP and TOKEN) are
demonstrated, along with most of the kinds
JavaCC regular expression, and also
local definitions (prefixed by a `\verb+#+'
in a TOKEN definition). See the JavaCC document
for full details. The syntax-definitions part 
of this file simply matches tokens and
prints them out. Notice that it is possible
to capture the token recognised in a Token object 
(here \verb+Token t+) and then to access and print
its kind (from the table \verb+tokenImage+ indexed
by \verb+Token+ field `\verb+kind+') and
the string that was matched (from \verb+Token+ field
`\verb+image+'). TOKENs which are defined as simple
strings (eg \verb+KEYTRUE+) will be printed as 
the string rather than the name (\verb+KEYTRUE+),
whereas for those with more complex definitions
the names are printed (eg  \verb+<IDENTIFIER>+)

Another example can be found in
\verb+/soi/sw/courses/daveb/IN2009/appel2.9+ (see the README file in the 
directory). 


\subsubsection*{Modifying the analyser}

(Part of the first coursework will ask a question something like this.)

Replace the regular expression
definition of the REAL token in the \verb+LexTest.jj+ file
to instead match signed real
numbers as written in Pascal.
Such numbers must contain a decimal point, and at least one
digit before and after the decimal point.
They may optionally be followed by a signed exponent
that begins with the letter `\verb"E"' and is followed
by a (possibly signed) integer.
In this notation, \verb"39.37", \verb"-6.336E4", \verb"0.894E-4" and
\verb"0.0" are legal, while \verb".36", \verb"4."
and \verb"+.7E6" are illegal.

\end{document}

