\documentclass{article}

\usepackage{a4wide}
\usepackage[breaklinks=true,
        colorlinks=true,urlcolor=blue,pdfpagemode=None]{hyperref}

\newcommand{\fname}[1]{\texttt{#1}}


\begin{document}

\subsection*{Language Processors -- Induction}

{\bf Note:} Read this through {\em before\/} logging in.

\medskip\noindent These sheets
introduce you to the Linux/Unix Java programming environment to 
be used on the module by getting you to compile and run a program.
You will be expected to be able to copy and 
manipulate program files and directories or web locations
named in lab sheets, and compile and run the programs, 
and then to modify them and compile and run them.
If you are not familiar with the Linux labs you should also get a copy of any
\emph{Computing Services} documentation available.


\subsection*{The Linux environment}

\subsubsection*{Logging in}

Linux/Unix labs contain PCs running Linux.
If you see a screensaver or a blank screen,
you can jiggle the mouse or hit
a key to get the machine's attention.

\subsubsection*{Logging Out}

\emph{Always remember}
to log out when you have finished working at a machine.
To logout from a Linux machine, press the K button
at the left of the control panel at the bottom of the screen, and
select {\sf Logout} from the menu. 

If you leave a machine without logging out you are putting your own work
at risk.

\subsubsection*{Working with the KDE windowing system under Linux}

Each computer is attached by a network to several file systems. Under
Linux, the fact that there are multiple file systems is something
you don't normally have to think about. All the available systems
just appear as different directories (folders) in a large
`tree' of directories.
(The tree is upside-down with its root at the top. The Root
Directory is called ``{\tt /}''.)

You need to become proficient at examining the contents of
directories (folders), and in copying and moving
directories and files around the file systems.
We will use the words `directory' and `folder'
to mean the same thing.

\begin{quote}
{\bf Please note}: click \emph{once}, not twice. If you are a Windows
user you will be used to double-clicking to make things happen. In KDE
you only click once. If you double click you may find you have done something
(like launching an application) twice. Don't worry, it's not usually fatal.
\end{quote}
You will notice that KDE has a Windows-like `taskbar' (the control panel)
at the bottom
of the screen with a K button 
on the bottom-left -- as on Windows, clicking on this
reveals a menu that launches applications etc.


\subsubsection*{Your CSD and SOI home directories}

When logged in as a Linux user you have a \emph{home directory}.
This is a place in the file system which belongs to you: the place where
you can create, modify and delete files and folders. When you start
the file manager (the button labelled with a small house icon on the control
panel is the easiest way to start it), your home directory is the one it will
display. You can also get the file manager to take you to your
home directory at any time simply by clicking on the home icon
on the task bar at the top.
Use the file manager to go to your home directory now.
Does it contain any files? Does it contain any hidden files?
(Hint: look on the {\sf View} menu.)

A slight complication (and a big advantage!)
for you as a student in the School of Informatics,
is that you have \emph{two} computing accounts: one provided by the
Computing Services Department (CSD) and one provided by
the School of Informatics (SOI). These two accounts give you access to
two distinct home directories on two distinct file systems. When you
are logged in to CSD machines as in the Unix labs
you are using your CSD account
and your home directory is (naturally) your CSD home directory.

Although the two home directories are distinct, they are not
completely isolated from each other.
From the CSD Linux machines your SOI home directory
is available as \fname{/soi/homes/}$<${\sc user}$>$, 
where $<${\sc user}$>$ is your login-name. 
This gives you access to a substantial
amount of file space in addition to your CSD quota.
Of course you can choose to do all your work there if you wish.
Use the file manager to go to your SOI home directory now.
Does it contain any files? Does it contain any hidden files?


\subsubsection*{Shells, and your directories}

You will need to be able to fire up and use a Unix shell 
window -- this is on the K menu, 
and is also one of the icons on the taskbar (the one with a 
terminal screen on it, marked with a shell). 

Fire one up now and then type
the command `\verb+ls+' inside the window. 
You will see listed the contents of your CSD home directory. 
You can change to your SOI home directory with the 
command:
\begin{quote}
{\tt cd} \fname{/soi/homes/}$<${\sc user}$>$
\end{quote}
You will find it very useful to learn about how to
move around and see the directory hierarchy using
the commands \verb+cd+ and \verb+ls+.


\subsubsection*{Browsers and Web pages}

On the Linux machines there are two main ways to browse web pages:
(1) using the file manager, (2) using Mozilla or one of the other 
Web browsers.
Here we will focus on using the file manager but if you would prefer to
use Mozilla you can. 

To view a web page in the file manager you can type the url into
the box along the top (you can usually leave off the {\tt http://}
part: the file manager will put that bit in for you). Since urls are
often very long and typing them is very boring, you will want to
bookmark any page that you frequently visit. Then you can visit the
page next time by simply clicking on the bookmark. To add and use
bookmarks in the file manager, use the {\sf Bookmarks} menu at the top.

%\subsubsection*{Disabling CleanStart}
%
%Your KDE configuration includes a feature which causes it to reset
%%itself to the default settings each time you log in. This is designed to
%help beginners who may accidentally mess up their configuration.
%%Unfortunately, it also causes you to lose your file manager bookmarks
%and other things which you need to keep, so you must disable this
%feature before going any further.
%\begin{quote}
    %Use the file manager to go to your home directory.
    %Open the folder called \fname{Desktop}.
    %In the \fname{Desktop} folder, open the folder
    %called \fname{Autostart}.
    %You will see a file called \fname{CleanStart.kdelnk}.
    %Delete this file by right-clicking on it and selecting {\sf Delete}.
%\end{quote}
%You have now disabled the CleanStart feature. If you want to re-enable
%the CleanStart feature, click on the {\bf K} button
%and use the {\sf System} sub-menu.
%

\subsection*{Java environment and programming}

The Java environment we will use is Suns JDK (Java Development
Kit, now also known as the SDK), which can be considered to be the reference 
Java environment, and provides (amongst other things)
a compiler, interpreter and debugger. These
are not integrated into a closed environment, so
you need to use at least the Unix file system as above, and 
a Unix editor to be able to create and modify programs (see below),
and learn the compiler and other commands. 


\subsubsection*{Getting some Java files to work with}

Fire up a Unix shell window,
move using the `\verb+cd+' command
to where you want to work in 
the filesystem, and type the command:
\begin{verbatim}
   cp -R /soi/sw/courses/daveb/javatest ./Shapes
\end{verbatim}
This makes a complete copy
of a simple Java program and associated files in 
current working directory in a new directory called \verb+Shapes+.
Now do:
\begin{verbatim}
   cd Shapes
   ls
\end{verbatim}
and you will see the Java files.


\subsubsection*{Compiling and running Java programs on Unix}

To use Java, in a shell window
you must add module `\verb+java+' (this 
gives access to the Java JDK (also known as 
SDK) and some local Java libraries):
\begin{verbatim}
   module add java
\end{verbatim}
To use School of Informatics specific software you 
need to add the module `\verb+soi+':
\begin{verbatim}
   module add soi
\end{verbatim}
These can be combined into one command:
\begin{verbatim}
   module add java soi
\end{verbatim}
The Java compilation command is `\verb+javac+', which will compile 
single or multiple `\verb+.java+' files: 
\begin{verbatim}
   javac Scene.java
\end{verbatim}
or, to compile all Java source files in the current directory,
\begin{verbatim}
   javac *.java
\end{verbatim}
(remember you must be in the directory containing these
files and the `\verb+cd+' command is used to change directories).
\emph{Your first compilation will take a long time; subsequent ones
take much less time.}

The command `\verb+java+' executes a Java program;
you give it a class name as an argument (so typically
you use the name of the class containing your \verb+main()+):
\begin{verbatim}
   java ScalableScene
\end{verbatim}
When you run a program like this, any error or other messages will 
appear in the window from which you issued the \verb+java+ command.

Remember that the \verb+Ctrl-C+ command typed in a shell window
will kill off any program that you have started from that 
shell window (often written \verb+^C+ -- 
press the \verb+Ctrl+ and \verb+C+ keys at the same time). 

Now compile and run a Java program
(eg the one copied above, or one of your own). 

\subsubsection*{Editing programs}

Linux/Unix has a variety of program and text editors available.
Those who are experienced Linux/Unix
users will wish to use what they're familiar with, probably
\verb+emacs+ or \verb+vi+. I suggest others use \verb+kate+,
which does syntax colouring and is set up for Java and other
program editing. This is available as command \verb+kate+ from
a Unix shell window (do \verb+kate &+ or \verb+kate [filename]+), 
or from the K menu as Editor.

I suggest you now simply modify some trivial aspects of
one of your programs and make sure you can save and recompile and
run it. 

\subsubsection*{And, finally, where is my {\tt U:} drive stuff?}

Your CSD XP PC (CADDI) \verb+U:+ drive is accessible as the folder
\verb+My Documents+ from your home directory on the CSD Unix systems. 
You can also, however, use the FTP system
to transfer files {\em\/} between the CSD XP PC (CADDI) and the 
Unix systems (either to your CSD Unix home directory or to your 
SOI home directory).


\end{document}

