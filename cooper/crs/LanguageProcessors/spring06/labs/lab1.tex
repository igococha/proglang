\documentclass{article}

\usepackage{a4wide}
\usepackage[breaklinks=true,
        colorlinks=true,urlcolor=blue,pdfpagemode=None]{hyperref}

\newcommand{\fname}[1]{\texttt{#1}}


\begin{document}

\subsection*{Language Processors Lab 1}

{\bf Note:} Read this through {\em before\/} logging in.

\medskip\noindent The goal for this weeks 
lab is to see the Straightline programming language interpreter working
and to add the {\tt maxargs(Stm s)} function to it. 

\subsubsection*{Running the Straightline interpreter}

Fire up a Unix shell window. 
To get the environment ready type the command:
\begin{verbatim}
   module add java soi 
\end{verbatim}
Move to the directory in which you want to do
your IN2009 work and copy the Straightline program directory
with the command:
\begin{verbatim}
   cp -R /soi/sw/courses/daveb/IN2009/straightline .
\end{verbatim}
This makes a complete copy of my straightline interpreter program. 
You can see it with:
\begin{verbatim}
   cd straightline
   ls
\end{verbatim}
{\em Note: \/}All the directories in \verb+/soi/sw/courses/daveb/IN2009+ are
available through CitySpace/WebCT, as single files,
or as zipped folder (directory) hierarchies, so you can download them from 
there instead of copying them if you wish.

Now compile the program with:
\begin{verbatim}
   javac *.java
\end{verbatim}
And then run the program with:
\begin{verbatim}
   java interp
\end{verbatim}
The interpreter output for the program given
in Appel will be printed.
There are other programs in file \verb+prog.java+;
modify \verb+interp.java+ to run some of these
instead of \verb+prog+ (eg \verb+prog2+).

\subsubsection*{Modifying the interpreter}

Work out how to write the \verb+maxargs(Stm s)+
function (the \verb+length()+ function already
defined in \verb+interp.java+ will be useful).
You will need several functions to do the count for
each abstract syntax type (see the implementation
of \verb+interp()+).

Edit the file \verb+interp.java+ to include your 
implementation of \verb+maxargs()+, and then
test your implementation by recompilation and
running.

\end{document}

