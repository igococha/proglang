\documentclass{article}

\usepackage{a4wide}
\usepackage[breaklinks=true,
        colorlinks=true,urlcolor=blue,pdfpagemode=None]{hyperref}

\begin{document}
\thispagestyle{empty}

\section*{Language Processors Lab 4}

{\bf Note:} Read this through {\em before\/} logging in.

\medskip\noindent The goal for this
lab is to see some JavaCC parsers working 
that have embedded actions to do
calculations and build syntax trees, 
and how to traverse the trees. 


\subsection*{Expression evaluator}

Move to the directory in which you want to do
your work and copy my directory and set up the shell to 
use JavaCC etc with the commands:
\begin{verbatim}
   cp -R /soi/sw/courses/daveb/IN2009/ebnfcalc .
   cd ebnfcalc
   module add java soi javacc/3.2
\end{verbatim}
The directory contains a JavaCC specification (tokens and grammar) 
for a simple expression language in file \verb+Exp.jj+. 
It also contains actions in the grammar that turns the parser
into an evaluator for expressions.
The file \verb+Main.java+ contains a class Main 
with the \verb+main()+ method that sets 
everything up and calls the parser. 
You can create the evaluator with:
\begin{verbatim}
   javacc Exp.jj
   javac Main.java
\end{verbatim}
and then run it with:
\begin{verbatim}
   java Main
\end{verbatim}
If you type in a legal expression (eg \verb=3+4*5+9=), the expression
will be evaluated and the result printed out.
Read the \verb+Exp.jj+ and \verb+Main.java+ files and make sure
you understand how it works.

There is a BNF version of the expression evaluator at
\verb+/soi/sw/courses/daveb/IN2009/bnfcalc+. Copy it
and try it in the same way as above. It is a little different 
to the EBNF version, since the non-terminal rules have to take
parameters as well as deliver values in order to perform
the calculations.

\subsection*{Abstract syntax tree}

A version of the expression evaluator that builds abstract
syntax trees and traverses them to perform the
evaluation is at \verb+/soi/sw/courses/daveb/IN2009/ebnfastcalc+.
The \verb+Exp.jj+ file now contains actions in
the grammar that build syntax trees rather 
than perform evaluations.
The syntax tree classes are in the subdirectory \verb+syntaxtree+,
and each class has an \verb+eval()+ method that performs the
evaluations and recursively calls \verb+eval()+ on
any subtrees.  
Copy it and try it, and then read and understand the various
files.

\subsection*{Visitors}

A version of the expression evaluator that builds
abstract syntax trees and then traverses them using
the visitor pattern is at
\verb+/soi/sw/courses/daveb/IN2009/ebnfviscalc+.
The visitor classes are in subdirectory \verb+visitor+;
each syntax tree class now contains an \verb+accept()+
method for a visitor.
Copy it and try it, and then read and understand the various
files. 

This example is extended with a simple pretty printer for
the syntax trees in \\
\verb+/soi/sw/courses/daveb/IN2009/ebnfvisprint+.
Before you look at how I have done it (see file \verb+visitor/AstPrint.java+),
work out how you would do it.

\subsection*{MiniJava abstract syntax trees and visitors}

The MiniJava parser that constructs abstract syntax trees is at
\\
 \verb+/soi/sw/courses/daveb/IN2009/minijava/chap4+.
The \verb+README+ file explains how to compile and
run it.


\end{document}

